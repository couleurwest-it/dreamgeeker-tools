%% Generated by Sphinx.
\def\sphinxdocclass{report}
\documentclass[letterpaper,10pt,french]{sphinxmanual}
\ifdefined\pdfpxdimen
   \let\sphinxpxdimen\pdfpxdimen\else\newdimen\sphinxpxdimen
\fi \sphinxpxdimen=.75bp\relax

\PassOptionsToPackage{warn}{textcomp}
\usepackage[utf8]{inputenc}
\ifdefined\DeclareUnicodeCharacter
% support both utf8 and utf8x syntaxes
  \ifdefined\DeclareUnicodeCharacterAsOptional
    \def\sphinxDUC#1{\DeclareUnicodeCharacter{"#1}}
  \else
    \let\sphinxDUC\DeclareUnicodeCharacter
  \fi
  \sphinxDUC{00A0}{\nobreakspace}
  \sphinxDUC{2500}{\sphinxunichar{2500}}
  \sphinxDUC{2502}{\sphinxunichar{2502}}
  \sphinxDUC{2514}{\sphinxunichar{2514}}
  \sphinxDUC{251C}{\sphinxunichar{251C}}
  \sphinxDUC{2572}{\textbackslash}
\fi
\usepackage{cmap}
\usepackage[T1]{fontenc}
\usepackage{amsmath,amssymb,amstext}
\usepackage{babel}



\usepackage{times}
\expandafter\ifx\csname T@LGR\endcsname\relax
\else
% LGR was declared as font encoding
  \substitutefont{LGR}{\rmdefault}{cmr}
  \substitutefont{LGR}{\sfdefault}{cmss}
  \substitutefont{LGR}{\ttdefault}{cmtt}
\fi
\expandafter\ifx\csname T@X2\endcsname\relax
  \expandafter\ifx\csname T@T2A\endcsname\relax
  \else
  % T2A was declared as font encoding
    \substitutefont{T2A}{\rmdefault}{cmr}
    \substitutefont{T2A}{\sfdefault}{cmss}
    \substitutefont{T2A}{\ttdefault}{cmtt}
  \fi
\else
% X2 was declared as font encoding
  \substitutefont{X2}{\rmdefault}{cmr}
  \substitutefont{X2}{\sfdefault}{cmss}
  \substitutefont{X2}{\ttdefault}{cmtt}
\fi


\usepackage[Sonny]{fncychap}
\ChNameVar{\Large\normalfont\sffamily}
\ChTitleVar{\Large\normalfont\sffamily}
\usepackage{sphinx}

\fvset{fontsize=\small}
\usepackage{geometry}


% Include hyperref last.
\usepackage{hyperref}
% Fix anchor placement for figures with captions.
\usepackage{hypcap}% it must be loaded after hyperref.
% Set up styles of URL: it should be placed after hyperref.
\urlstyle{same}


\usepackage{sphinxmessages}




\title{dreamtools}
\date{déc. 20, 2020}
\release{1.0.1}
\author{dreamgeeker}
\newcommand{\sphinxlogo}{\vbox{}}
\renewcommand{\releasename}{Version}
\makeindex
\begin{document}

\ifdefined\shorthandoff
  \ifnum\catcode`\=\string=\active\shorthandoff{=}\fi
  \ifnum\catcode`\"=\active\shorthandoff{"}\fi
\fi

\pagestyle{empty}
\sphinxmaketitle
\pagestyle{plain}
\sphinxtableofcontents
\pagestyle{normal}
\phantomsection\label{\detokenize{index::doc}}



\chapter{Dreamtools}
\label{\detokenize{readme:dreamtools}}\label{\detokenize{readme::doc}}
Dreamtools est un outils d’aide au développement contenant une liste de fonction d’utilisation basique


\section{Installation}
\label{\detokenize{readme:installation}}
\begin{sphinxVerbatim}[commandchars=\\\{\}]
\PYGZdl{} pip install deamtools
\PYGZdl{} tools\PYGZhy{}install
\end{sphinxVerbatim}

Le répertoire de configuration “cfg” sera créé à la racine du projet.

\begin{sphinxadmonition}{warning}{Avertissement:}
Initiliser les données de l’application
\end{sphinxadmonition}


\section{Configuration}
\label{\detokenize{readme:configuration}}
\begin{sphinxVerbatim}[commandchars=\\\{\}]
\PYG{k+kn}{import} \PYG{n+nn}{toolbox}

\PYG{n}{app\PYGZus{}name} \PYG{o}{=} \PYG{l+s+s2}{\PYGZdq{}}\PYG{l+s+s2}{AMON\PYGZus{}APP}\PYG{l+s+s2}{\PYGZdq{}}   \PYG{c+c1}{\PYGZsh{}nom de votre application}
\PYG{n}{toolbox}\PYG{o}{.}\PYG{n}{config}\PYG{p}{(}\PYG{n}{app\PYGZus{}name}\PYG{p}{,} \PYG{n}{mode}\PYG{o}{=}\PYG{l+s+s1}{\PYGZsq{}}\PYG{l+s+s1}{DEBUG}\PYG{l+s+s1}{\PYGZsq{}}\PYG{p}{)}  \PYG{c+c1}{\PYGZsh{} par défaut mode =\PYGZsq{}PROD\PYGZsq{}}
\end{sphinxVerbatim}

\begin{sphinxadmonition}{warning}{Avertissement:}
Le paquet comprend un module de cryptage non supporté par Winddows
\end{sphinxadmonition}


\section{Crédits}
\label{\detokenize{readme:credits}}
Conçut par Dreamgeerker


\chapter{Librairie de fonctions}
\label{\detokenize{modules/tools:module-toolbox.tools}}\label{\detokenize{modules/tools:librairie-de-fonctions}}\label{\detokenize{modules/tools::doc}}\index{module@\spxentry{module}!toolbox.tools@\spxentry{toolbox.tools}}\index{toolbox.tools@\spxentry{toolbox.tools}!module@\spxentry{module}}

\section{Module de fonctions basiques}
\label{\detokenize{modules/tools:module-de-fonctions-basiques}}
Liste de fonctions utiles

pathfile : toolbox/tools


\subsection{Constantes globales}
\label{\detokenize{modules/tools:constantes-globales}}
\begin{sphinxadmonition}{note}{Note:}\begin{itemize}
\item {} 
RGX\_ACCENTS = “àâäãéèêëîïìôöòõùüûÿñç”

\item {} 
RGX\_EMAIL = Expression reguliere email

\item {} 
RGX\_PUNCT = Caractere speciaux autorisé pour mot de passe

\item {} 
RGX\_PWD = Expression régulière pour un mot de passe de 8 à 12 avec un car.Special/une Majuscule/Une minuscule

\item {} 
RGX\_PHONE = Expression réguliere remative à un numéro de téléphon

\item {} 
RGX\_URL = expression reguliere pour unr

\end{itemize}
\end{sphinxadmonition}


\subsection{Fonctions}
\label{\detokenize{modules/tools:fonctions}}\index{add\_list() (dans le module toolbox.tools)@\spxentry{add\_list()}\spxextra{dans le module toolbox.tools}}

\begin{fulllineitems}
\phantomsection\label{\detokenize{modules/tools:toolbox.tools.add_list}}\pysiglinewithargsret{\sphinxbfcode{\sphinxupquote{add\_list}}}{\emph{\DUrole{n}{v}}, \emph{\DUrole{n}{ll}}}{}
Ajout d’un item dans une liste avec gestion des doublons
\begin{quote}\begin{description}
\item[{Paramètres}] \leavevmode\begin{itemize}
\item {} 
\sphinxstyleliteralstrong{\sphinxupquote{v}} (\sphinxstyleliteralemphasis{\sphinxupquote{str}}) \textendash{} valeur à ajouter

\item {} 
\sphinxstyleliteralstrong{\sphinxupquote{l}} (\sphinxstyleliteralemphasis{\sphinxupquote{list}}) \textendash{} liste

\end{itemize}

\end{description}\end{quote}

\end{fulllineitems}

\index{addhex() (dans le module toolbox.tools)@\spxentry{addhex()}\spxextra{dans le module toolbox.tools}}

\begin{fulllineitems}
\phantomsection\label{\detokenize{modules/tools:toolbox.tools.addhex}}\pysiglinewithargsret{\sphinxbfcode{\sphinxupquote{addhex}}}{\emph{\DUrole{n}{h}}, \emph{\DUrole{n}{v}}}{}
Additionne une valeur hexadécimal
\begin{quote}\begin{description}
\item[{Paramètres}] \leavevmode\begin{itemize}
\item {} 
\sphinxstyleliteralstrong{\sphinxupquote{h}} (\sphinxstyleliteralemphasis{\sphinxupquote{str}}) \textendash{} valeur hexadécimal

\item {} 
\sphinxstyleliteralstrong{\sphinxupquote{v}} (\sphinxstyleliteralemphasis{\sphinxupquote{str}}) \textendash{} valeur entière à ajouter

\end{itemize}

\item[{Renvoie}] \leavevmode
valeur additionné en hexedécimal

\item[{Example}] \leavevmode
\begin{sphinxVerbatim}[commandchars=\\\{\}]
\PYG{g+gp}{\PYGZgt{}\PYGZgt{}\PYGZgt{} }\PYG{n}{hx} \PYG{o}{=} \PYG{l+s+s1}{\PYGZsq{}}\PYG{l+s+s1}{0x129}\PYG{l+s+s1}{\PYGZsq{}}
\PYG{g+gp}{\PYGZgt{}\PYGZgt{}\PYGZgt{} }\PYG{n}{addhex}\PYG{p}{(}\PYG{n}{hx}\PYG{p}{,} \PYG{l+m+mi}{2}\PYG{p}{)}
\PYG{g+go}{0x12b}
\end{sphinxVerbatim}

\end{description}\end{quote}

\end{fulllineitems}

\index{aleatoire() (dans le module toolbox.tools)@\spxentry{aleatoire()}\spxextra{dans le module toolbox.tools}}

\begin{fulllineitems}
\phantomsection\label{\detokenize{modules/tools:toolbox.tools.aleatoire}}\pysiglinewithargsret{\sphinxbfcode{\sphinxupquote{aleatoire}}}{\emph{\DUrole{n}{end}}, \emph{\DUrole{n}{s}\DUrole{o}{=}\DUrole{default_value}{1}}}{}
Génération d’un nombre aléatoire entre {[}1\sphinxhyphen{}end{]} =\textgreater{} end caractère
\begin{quote}\begin{description}
\item[{Paramètres}] \leavevmode\begin{itemize}
\item {} 
\sphinxstyleliteralstrong{\sphinxupquote{end}} (\sphinxstyleliteralemphasis{\sphinxupquote{int}}) \textendash{} valeur maximal (paut indiquer la taille si s=1)

\item {} 
\sphinxstyleliteralstrong{\sphinxupquote{s}} \textendash{} valeur de départ, default to 1

\end{itemize}

\item[{Renvoie}] \leavevmode
Un chiffre aléatoire

\item[{Exemple}] \leavevmode
\begin{sphinxVerbatim}[commandchars=\\\{\}]
\PYG{g+gp}{\PYGZgt{}\PYGZgt{}\PYGZgt{} }\PYG{n}{aleatoire} \PYG{p}{(}\PYG{l+m+mi}{5}\PYG{p}{)}
\PYG{g+go}{1 : Renvoie un chiffre entre 1 et 5}
\PYG{g+gp}{\PYGZgt{}\PYGZgt{}\PYGZgt{} }\PYG{n}{aleatoire} \PYG{p}{(}\PYG{l+m+mi}{5}\PYG{p}{,}\PYG{l+m+mi}{3}\PYG{p}{)}
\PYG{g+go}{1 : Renvoie un chiffre entre 3 et 5}
\PYG{g+gp}{\PYGZgt{}\PYGZgt{}\PYGZgt{} }\PYG{l+m+mi}{4}
\end{sphinxVerbatim}

\end{description}\end{quote}

\end{fulllineitems}

\index{check\_password() (dans le module toolbox.tools)@\spxentry{check\_password()}\spxextra{dans le module toolbox.tools}}

\begin{fulllineitems}
\phantomsection\label{\detokenize{modules/tools:toolbox.tools.check_password}}\pysiglinewithargsret{\sphinxbfcode{\sphinxupquote{check\_password}}}{\emph{\DUrole{n}{s}}}{}
Vérifie que la syntaxe d’une chaine répond au critère d’un mot de passe
\begin{quote}\begin{description}
\item[{Conditions}] \leavevmode\begin{itemize}
\item {} 
Une majuscule

\item {} 
Une minuscule

\item {} 
Un chiffre

\item {} 
Un carectère spécial (@\#!?\$\&\sphinxhyphen{}\_ autorisé )

\end{itemize}

\item[{Paramètres}] \leavevmode
\sphinxstyleliteralstrong{\sphinxupquote{s}} (\sphinxstyleliteralemphasis{\sphinxupquote{str}}) \textendash{} chaine à vérifier

\item[{Return bool}] \leavevmode
True si la chaine est valide

\end{description}\end{quote}

\end{fulllineitems}

\index{clean\_allspace() (dans le module toolbox.tools)@\spxentry{clean\_allspace()}\spxextra{dans le module toolbox.tools}}

\begin{fulllineitems}
\phantomsection\label{\detokenize{modules/tools:toolbox.tools.clean_allspace}}\pysiglinewithargsret{\sphinxbfcode{\sphinxupquote{clean\_allspace}}}{\emph{\DUrole{n}{ch}}, \emph{\DUrole{n}{very\_all}\DUrole{o}{=}\DUrole{default_value}{True}}}{}
Nettoyage de tous les espace et carateres vides
\begin{quote}\begin{description}
\item[{Paramètres}] \leavevmode\begin{itemize}
\item {} 
\sphinxstyleliteralstrong{\sphinxupquote{ch}} (\sphinxstyleliteralemphasis{\sphinxupquote{str}}) \textendash{} Chaine à nettoyer

\item {} 
\sphinxstyleliteralstrong{\sphinxupquote{very\_all}} (\sphinxstyleliteralemphasis{\sphinxupquote{bool}}) \textendash{} caractère vide aussi, True (False = Espaces uniquement)

\end{itemize}

\item[{Exemple}] \leavevmode
\begin{sphinxVerbatim}[commandchars=\\\{\}]
\PYG{g+gp}{\PYGZgt{}\PYGZgt{}\PYGZgt{} }\PYG{n}{chaine} \PYG{o}{=} \PYG{l+s+s1}{\PYGZsq{}}\PYG{l+s+s1}{Se  réveiller au matin        de sa destiné !}\PYG{l+s+s1}{\PYGZsq{}}
\PYG{g+gp}{\PYGZgt{}\PYGZgt{}\PYGZgt{} }\PYG{n}{clean\PYGZus{}allspace} \PYG{p}{(}\PYG{n}{chaine}\PYG{p}{)}
\PYG{g+go}{\PYGZsq{}Seréveilleraumatindesadestiné!\PYGZsq{}}
\end{sphinxVerbatim}

\end{description}\end{quote}

\end{fulllineitems}

\index{clean\_coma() (dans le module toolbox.tools)@\spxentry{clean\_coma()}\spxextra{dans le module toolbox.tools}}

\begin{fulllineitems}
\phantomsection\label{\detokenize{modules/tools:toolbox.tools.clean_coma}}\pysiglinewithargsret{\sphinxbfcode{\sphinxupquote{clean\_coma}}}{\emph{\DUrole{n}{ch}}, \emph{\DUrole{n}{w\_punk}\DUrole{o}{=}\DUrole{default_value}{False}}}{}
Supprime les accents/caractères spéciaux du texte source en respectant la casse
\begin{quote}\begin{description}
\item[{Paramètres}] \leavevmode\begin{itemize}
\item {} 
\sphinxstyleliteralstrong{\sphinxupquote{ch}} \textendash{} Chaine de caractere à « nettoyer »

\item {} 
\sphinxstyleliteralstrong{\sphinxupquote{w\_punk}} \textendash{} indique si la punctuation est à nettoyer ou pas (suppression)

\end{itemize}

\item[{Exemple}] \leavevmode
\begin{sphinxVerbatim}[commandchars=\\\{\}]
\PYG{g+gp}{\PYGZgt{}\PYGZgt{}\PYGZgt{} }\PYG{n}{s} \PYG{o}{=} \PYG{l+s+s1}{\PYGZsq{}}\PYG{l+s+s1}{Se  réveiller au matin    de sa destiné !!}\PYG{l+s+s1}{\PYGZsq{}}
\PYG{g+gp}{\PYGZgt{}\PYGZgt{}\PYGZgt{} }\PYG{n}{clean\PYGZus{}coma} \PYG{p}{(}\PYG{n}{s}\PYG{p}{)}
\PYG{g+go}{\PYGZsq{}Se seveiller au matin (ou pas) de sa destine !!\PYGZsq{}\PYGZsq{}}
\PYG{g+gp}{\PYGZgt{}\PYGZgt{}\PYGZgt{} }\PYG{n}{clean\PYGZus{}coma} \PYG{p}{(}\PYG{n}{s}\PYG{p}{,} \PYG{k+kc}{True}\PYG{p}{)}
\PYG{g+go}{\PYGZsq{}Se reveiller au matin ou pas de sa destine\PYGZsq{}}
\end{sphinxVerbatim}

\end{description}\end{quote}

\end{fulllineitems}

\index{clean\_dir() (dans le module toolbox.tools)@\spxentry{clean\_dir()}\spxextra{dans le module toolbox.tools}}

\begin{fulllineitems}
\phantomsection\label{\detokenize{modules/tools:toolbox.tools.clean_dir}}\pysiglinewithargsret{\sphinxbfcode{\sphinxupquote{clean\_dir}}}{\emph{\DUrole{n}{directory}}, \emph{\DUrole{n}{pattern}\DUrole{o}{=}\DUrole{default_value}{\textquotesingle{}*\textquotesingle{}}}}{}
Supprimes tous les élements d’un repertoire
\begin{quote}\begin{description}
\item[{Paramètres}] \leavevmode\begin{itemize}
\item {} 
\sphinxstyleliteralstrong{\sphinxupquote{directory}} (\sphinxstyleliteralemphasis{\sphinxupquote{str}}) \textendash{} chemin du repertoire

\item {} 
\sphinxstyleliteralstrong{\sphinxupquote{pattern}} (\sphinxstyleliteralemphasis{\sphinxupquote{string}}) \textendash{} patter des fichier à supprimer (filtre)

\end{itemize}

\item[{Return int}] \leavevmode
nombre de fichier supprimer

\end{description}\end{quote}

\end{fulllineitems}

\index{clean\_master() (dans le module toolbox.tools)@\spxentry{clean\_master()}\spxextra{dans le module toolbox.tools}}

\begin{fulllineitems}
\phantomsection\label{\detokenize{modules/tools:toolbox.tools.clean_master}}\pysiglinewithargsret{\sphinxbfcode{\sphinxupquote{clean\_master}}}{\emph{\DUrole{n}{ch}}}{}
Supprime les accents, caractères spéciaux et espace du texte source
\begin{quote}\begin{description}
\item[{Paramètres}] \leavevmode
\sphinxstyleliteralstrong{\sphinxupquote{ch}} (\sphinxstyleliteralemphasis{\sphinxupquote{str}}) \textendash{} Chaine de caractere à « nettoyer »

\item[{Return str}] \leavevmode
chaine sans accents:car. spéciaux ni espace en minuscule

\item[{Exemple}] \leavevmode
\begin{sphinxVerbatim}[commandchars=\\\{\}]
\PYG{g+gp}{\PYGZgt{}\PYGZgt{}\PYGZgt{} }\PYG{n}{s} \PYG{o}{=} \PYG{l+s+s1}{\PYGZsq{}}\PYG{l+s+s1}{Se  réveiller au matin  (ou pas) de sa destiné !}\PYG{l+s+s1}{\PYGZsq{}}
\PYG{g+gp}{\PYGZgt{}\PYGZgt{}\PYGZgt{} }\PYG{n}{clean\PYGZus{}master} \PYG{p}{(}\PYG{n}{s}\PYG{p}{)}
\PYG{g+go}{\PYGZsq{}sereveilleraumatinoupasdesadestine}
\end{sphinxVerbatim}

\end{description}\end{quote}

\end{fulllineitems}

\index{clean\_space() (dans le module toolbox.tools)@\spxentry{clean\_space()}\spxextra{dans le module toolbox.tools}}

\begin{fulllineitems}
\phantomsection\label{\detokenize{modules/tools:toolbox.tools.clean_space}}\pysiglinewithargsret{\sphinxbfcode{\sphinxupquote{clean\_space}}}{\emph{\DUrole{n}{ch}}}{}
Nettoyage des espaces « superflus »
\begin{itemize}
\item {} 
Espaces à gouche et à droite supprimés

\item {} 
Répétition d’espace réduit

\end{itemize}
\begin{quote}\begin{description}
\item[{Exemple}] \leavevmode
\begin{sphinxVerbatim}[commandchars=\\\{\}]
\PYG{g+gp}{\PYGZgt{}\PYGZgt{}\PYGZgt{} }\PYG{n}{chaine} \PYG{o}{=} \PYG{l+s+s1}{\PYGZsq{}}\PYG{l+s+s1}{Se  réveiller au matin        de sa destiné    ! !           }\PYG{l+s+s1}{\PYGZsq{}}
\PYG{g+gp}{\PYGZgt{}\PYGZgt{}\PYGZgt{} }\PYG{n}{clean\PYGZus{}space} \PYG{p}{(}\PYG{n}{chaine}\PYG{p}{)}
\PYG{g+go}{\PYGZsq{}Se réveiller au matin  de sa destiné ! !\PYGZsq{}}
\end{sphinxVerbatim}

\end{description}\end{quote}

\end{fulllineitems}

\index{code\_maker() (dans le module toolbox.tools)@\spxentry{code\_maker()}\spxextra{dans le module toolbox.tools}}

\begin{fulllineitems}
\phantomsection\label{\detokenize{modules/tools:toolbox.tools.code_maker}}\pysiglinewithargsret{\sphinxbfcode{\sphinxupquote{code\_maker}}}{\emph{\DUrole{n}{i\_size}\DUrole{o}{=}\DUrole{default_value}{4}}}{}
Génération d’une chaine aléatoire composé de lettre et de chiffres
\begin{quote}\begin{description}
\item[{Paramètres}] \leavevmode
\sphinxstyleliteralstrong{\sphinxupquote{i\_size}} (\sphinxstyleliteralemphasis{\sphinxupquote{int}}) \textendash{} taille du code

\item[{Rtype str}] \leavevmode
\end{description}\end{quote}

\end{fulllineitems}

\index{comphex() (dans le module toolbox.tools)@\spxentry{comphex()}\spxextra{dans le module toolbox.tools}}

\begin{fulllineitems}
\phantomsection\label{\detokenize{modules/tools:toolbox.tools.comphex}}\pysiglinewithargsret{\sphinxbfcode{\sphinxupquote{comphex}}}{\emph{\DUrole{n}{hx\_a}}, \emph{\DUrole{n}{hx\_b}}}{}
Compare deux valeurs  hexadécimales
\begin{quote}\begin{description}
\item[{Paramètres}] \leavevmode\begin{itemize}
\item {} 
\sphinxstyleliteralstrong{\sphinxupquote{hx\_a}} (\sphinxstyleliteralemphasis{\sphinxupquote{str}}) \textendash{} 

\item {} 
\sphinxstyleliteralstrong{\sphinxupquote{hx\_b}} (\sphinxstyleliteralemphasis{\sphinxupquote{str}}) \textendash{} 

\end{itemize}

\item[{Return int}] \leavevmode\begin{itemize}
\item {} 
0 : hx\_a == hx\_b

\item {} 
1 : hx\_a \textgreater{} hx\_b

\item {} 
\sphinxhyphen{}1 : hx\_a \textless{} hx\_b

\end{itemize}

\end{description}\end{quote}

\end{fulllineitems}

\index{dictlist() (dans le module toolbox.tools)@\spxentry{dictlist()}\spxextra{dans le module toolbox.tools}}

\begin{fulllineitems}
\phantomsection\label{\detokenize{modules/tools:toolbox.tools.dictlist}}\pysiglinewithargsret{\sphinxbfcode{\sphinxupquote{dictlist}}}{\emph{\DUrole{n}{k}}, \emph{\DUrole{n}{v}}, \emph{\DUrole{n}{d}}}{}
Ajout d’un valeur dans une liste d’un dictionnaire
\begin{quote}\begin{description}
\item[{Paramètres}] \leavevmode\begin{itemize}
\item {} 
\sphinxstyleliteralstrong{\sphinxupquote{k}} (\sphinxstyleliteralemphasis{\sphinxupquote{str}}) \textendash{} clé dictionnaire

\item {} 
\sphinxstyleliteralstrong{\sphinxupquote{v}} \textendash{} valeur à ajouter

\item {} 
\sphinxstyleliteralstrong{\sphinxupquote{list}}\sphinxstyleliteralstrong{\sphinxupquote{{[}}}\sphinxstyleliteralstrong{\sphinxupquote{{]}}}\sphinxstyleliteralstrong{\sphinxupquote{{]} }}\sphinxstyleliteralstrong{\sphinxupquote{p\_dic}} (\sphinxstyleliteralemphasis{\sphinxupquote{dict}}\sphinxstyleliteralemphasis{\sphinxupquote{{[}}}\sphinxstyleliteralemphasis{\sphinxupquote{str}}\sphinxstyleliteralemphasis{\sphinxupquote{,}}) \textendash{} dictionnaire

\end{itemize}

\item[{Exemple}] \leavevmode
\begin{sphinxVerbatim}[commandchars=\\\{\}]
\PYG{g+gp}{\PYGZgt{}\PYGZgt{}\PYGZgt{} }\PYG{n}{dictionnaire}\PYG{o}{=} \PYG{p}{\PYGZob{}}\PYG{p}{\PYGZcb{}}
\PYG{g+gp}{\PYGZgt{}\PYGZgt{}\PYGZgt{} }\PYG{n}{dictlist}\PYG{p}{(}\PYG{l+s+s1}{\PYGZsq{}}\PYG{l+s+s1}{printemps}\PYG{l+s+s1}{\PYGZsq{}}\PYG{p}{,} \PYG{l+s+s1}{\PYGZsq{}}\PYG{l+s+s1}{mar}\PYG{l+s+s1}{\PYGZsq{}}\PYG{p}{,} \PYG{n}{dictionnaire}\PYG{p}{)}
\PYG{g+go}{dictionnaire\PYGZob{}\PYGZsq{}printemps\PYGZsq{}, [\PYGZsq{}mars\PYGZsq{}]\PYGZcb{}}
\PYG{g+gp}{\PYGZgt{}\PYGZgt{}\PYGZgt{} }\PYG{n}{dictlist}\PYG{p}{(}\PYG{l+s+s1}{\PYGZsq{}}\PYG{l+s+s1}{printemps}\PYG{l+s+s1}{\PYGZsq{}}\PYG{p}{,} \PYG{l+s+s1}{\PYGZsq{}}\PYG{l+s+s1}{avril}\PYG{l+s+s1}{\PYGZsq{}}\PYG{p}{,} \PYG{n}{dictionnaire}\PYG{p}{)}
\PYG{g+go}{dictionnaire\PYGZob{}\PYGZsq{}printemps\PYGZsq{}, [\PYGZsq{}mars\PYGZsq{}, \PYGZsq{}\PYGZsq{}avril\PYGZsq{}]\PYGZcb{}}
\PYG{g+gp}{\PYGZgt{}\PYGZgt{}\PYGZgt{} }\PYG{n}{dictlist}\PYG{p}{(}\PYG{l+s+s1}{\PYGZsq{}}\PYG{l+s+s1}{printemps}\PYG{l+s+s1}{\PYGZsq{}}\PYG{p}{,} \PYG{l+s+s1}{\PYGZsq{}}\PYG{l+s+s1}{mars}\PYG{l+s+s1}{\PYGZsq{}}\PYG{p}{,} \PYG{n}{dictionnaire}\PYG{p}{)}
\PYG{g+go}{dictionnaire\PYGZob{}\PYGZsq{}printemps\PYGZsq{}, [\PYGZsq{}mars\PYGZsq{}, \PYGZsq{}\PYGZsq{}avril\PYGZsq{}]\PYGZcb{}}
\end{sphinxVerbatim}

\end{description}\end{quote}

\end{fulllineitems}

\index{dir\_parent() (dans le module toolbox.tools)@\spxentry{dir\_parent()}\spxextra{dans le module toolbox.tools}}

\begin{fulllineitems}
\phantomsection\label{\detokenize{modules/tools:toolbox.tools.dir_parent}}\pysiglinewithargsret{\sphinxbfcode{\sphinxupquote{dir\_parent}}}{\emph{\DUrole{n}{path}}}{}
Renvoie du repertoire parent
\begin{quote}\begin{description}
\item[{Paramètres}] \leavevmode
\sphinxstyleliteralstrong{\sphinxupquote{path}} (\sphinxstyleliteralemphasis{\sphinxupquote{str}}) \textendash{} repertoire

\item[{Type renvoyé}] \leavevmode
str

\end{description}\end{quote}

\end{fulllineitems}

\index{dir\_parser() (dans le module toolbox.tools)@\spxentry{dir\_parser()}\spxextra{dans le module toolbox.tools}}

\begin{fulllineitems}
\phantomsection\label{\detokenize{modules/tools:toolbox.tools.dir_parser}}\pysiglinewithargsret{\sphinxbfcode{\sphinxupquote{dir\_parser}}}{\emph{\DUrole{n}{directory}}, \emph{\DUrole{n}{pattern}\DUrole{o}{=}\DUrole{default_value}{\textquotesingle{}*\textquotesingle{}}}}{}
Récupération des fichiers d’un répertoire
\begin{quote}\begin{description}
\item[{Paramètres}] \leavevmode\begin{itemize}
\item {} 
\sphinxstyleliteralstrong{\sphinxupquote{directory}} (\sphinxstyleliteralemphasis{\sphinxupquote{str}}) \textendash{} repertoire

\item {} 
\sphinxstyleliteralstrong{\sphinxupquote{pattern}} (\sphinxstyleliteralemphasis{\sphinxupquote{str}}) \textendash{} “*” pour tous type de fichier par défaut

\end{itemize}

\item[{Exemple}] \leavevmode
\begin{sphinxVerbatim}[commandchars=\\\{\}]
\PYG{g+gp}{\PYGZgt{}\PYGZgt{}\PYGZgt{} }\PYG{n}{directory} \PYG{o}{=} \PYG{l+s+s1}{\PYGZsq{}}\PYG{l+s+s1}{C:}\PYG{l+s+s1}{\PYGZbs{}}\PYG{l+s+s1}{Users}\PYG{l+s+s1}{\PYGZbs{}}\PYG{l+s+s1}{public}\PYG{l+s+s1}{\PYGZbs{}}\PYG{l+s+s1}{Documents}\PYG{l+s+s1}{\PYGZsq{}}
\PYG{g+gp}{\PYGZgt{}\PYGZgt{}\PYGZgt{} }\PYG{n}{pattern}\PYG{o}{=}\PYG{l+s+s1}{\PYGZsq{}}\PYG{l+s+s1}{*.txt}\PYG{l+s+s1}{\PYGZsq{}}
\PYG{g+gp}{\PYGZgt{}\PYGZgt{}\PYGZgt{} }\PYG{k}{for} \PYG{n}{filename}\PYG{p}{,} \PYG{n}{path\PYGZus{}file} \PYG{o+ow}{in} \PYG{n}{dir\PYGZus{}parser}\PYG{p}{(}\PYG{n}{directory}\PYG{p}{,} \PYG{n}{pattern}\PYG{p}{)}\PYG{p}{:}
\PYG{g+gp}{... }   \PYG{n+nb}{print}\PYG{p}{(}\PYG{n}{path\PYGZus{}file}\PYG{p}{)}
\PYG{g+go}{\PYGZsq{}C:\PYGZbs{}Users\PYGZbs{}public\PYGZbs{}Documents\PYGZbs{}fichier.txt\PYGZsq{}}
\PYG{g+go}{\PYGZsq{}C:\PYGZbs{}Users\PYGZbs{}public\PYGZbs{}Documents\PYGZbs{}autre\PYGZus{}fichier.txt\PYGZsq{}}
\end{sphinxVerbatim}

\end{description}\end{quote}

\end{fulllineitems}

\index{dir\_projet() (dans le module toolbox.tools)@\spxentry{dir\_projet()}\spxextra{dans le module toolbox.tools}}

\begin{fulllineitems}
\phantomsection\label{\detokenize{modules/tools:toolbox.tools.dir_projet}}\pysiglinewithargsret{\sphinxbfcode{\sphinxupquote{dir\_projet}}}{}{}
Répertoire pour le fichier en cours
\begin{quote}\begin{description}
\item[{Type renvoyé}] \leavevmode
str

\end{description}\end{quote}

\end{fulllineitems}

\index{dir\_worked() (dans le module toolbox.tools)@\spxentry{dir\_worked()}\spxextra{dans le module toolbox.tools}}

\begin{fulllineitems}
\phantomsection\label{\detokenize{modules/tools:toolbox.tools.dir_worked}}\pysiglinewithargsret{\sphinxbfcode{\sphinxupquote{dir\_worked}}}{}{}
renvoie du repertoire d’execution

\end{fulllineitems}

\index{file\_exists() (dans le module toolbox.tools)@\spxentry{file\_exists()}\spxextra{dans le module toolbox.tools}}

\begin{fulllineitems}
\phantomsection\label{\detokenize{modules/tools:toolbox.tools.file_exists}}\pysiglinewithargsret{\sphinxbfcode{\sphinxupquote{file\_exists}}}{\emph{\DUrole{n}{fp}}}{}
Vérifie l’existance d’un fichier
\begin{quote}\begin{description}
\item[{Paramètres}] \leavevmode
\sphinxstyleliteralstrong{\sphinxupquote{fp}} (\sphinxstyleliteralemphasis{\sphinxupquote{str}}) \textendash{} filepath

\item[{Rtype bool}] \leavevmode
\end{description}\end{quote}

\end{fulllineitems}

\index{file\_ext() (dans le module toolbox.tools)@\spxentry{file\_ext()}\spxextra{dans le module toolbox.tools}}

\begin{fulllineitems}
\phantomsection\label{\detokenize{modules/tools:toolbox.tools.file_ext}}\pysiglinewithargsret{\sphinxbfcode{\sphinxupquote{file\_ext}}}{\emph{\DUrole{n}{ps\_file}}}{}
Retrourne l’extension d’un fichier
\begin{quote}\begin{description}
\item[{Paramètres}] \leavevmode
\sphinxstyleliteralstrong{\sphinxupquote{ps\_file}} \textendash{} 

\item[{Renvoie}] \leavevmode
Extension de fichier

\end{description}\end{quote}

\end{fulllineitems}

\index{inttohex() (dans le module toolbox.tools)@\spxentry{inttohex()}\spxextra{dans le module toolbox.tools}}

\begin{fulllineitems}
\phantomsection\label{\detokenize{modules/tools:toolbox.tools.inttohex}}\pysiglinewithargsret{\sphinxbfcode{\sphinxupquote{inttohex}}}{\emph{\DUrole{n}{v}}}{}
Conversion d’une valeur en hexadécimal
\begin{quote}\begin{description}
\item[{Paramètres}] \leavevmode
\sphinxstyleliteralstrong{\sphinxupquote{v}} (\sphinxstyleliteralemphasis{\sphinxupquote{int}}) \textendash{} nombre à convertir

\item[{Renvoie}] \leavevmode
valeur en hexadécimal

\item[{Type renvoyé}] \leavevmode
str

\end{description}\end{quote}

\end{fulllineitems}

\index{makedirs() (dans le module toolbox.tools)@\spxentry{makedirs()}\spxextra{dans le module toolbox.tools}}

\begin{fulllineitems}
\phantomsection\label{\detokenize{modules/tools:toolbox.tools.makedirs}}\pysiglinewithargsret{\sphinxbfcode{\sphinxupquote{makedirs}}}{\emph{\DUrole{n}{path}}}{}
Création du répertoire données
\begin{quote}\begin{description}
\item[{Paramètres}] \leavevmode
\sphinxstyleliteralstrong{\sphinxupquote{path}} \textendash{} chemin du répertoire à créer

\item[{Rtype bool}] \leavevmode
\end{description}\end{quote}

\end{fulllineitems}

\index{path\_build() (dans le module toolbox.tools)@\spxentry{path\_build()}\spxextra{dans le module toolbox.tools}}

\begin{fulllineitems}
\phantomsection\label{\detokenize{modules/tools:toolbox.tools.path_build}}\pysiglinewithargsret{\sphinxbfcode{\sphinxupquote{path\_build}}}{\emph{\DUrole{n}{directory}}, \emph{\DUrole{n}{ps\_complement}}}{}
Construction d’un pathfile
\begin{quote}\begin{description}
\item[{Paramètres}] \leavevmode\begin{itemize}
\item {} 
\sphinxstyleliteralstrong{\sphinxupquote{directory}} (\sphinxstyleliteralemphasis{\sphinxupquote{str}}) \textendash{} repertoire

\item {} 
\sphinxstyleliteralstrong{\sphinxupquote{ps\_complement}} (\sphinxstyleliteralemphasis{\sphinxupquote{str}}) \textendash{} complement permettant de generer le chemin

\end{itemize}

\item[{Type renvoyé}] \leavevmode
str

\item[{Exemple}] \leavevmode
\begin{sphinxVerbatim}[commandchars=\\\{\}]
\PYG{g+gp}{\PYGZgt{}\PYGZgt{}\PYGZgt{} }\PYG{n}{path} \PYG{o}{=} \PYG{l+s+s1}{\PYGZsq{}}\PYG{l+s+s1}{c:}\PYG{l+s+s1}{\PYGZbs{}}\PYG{l+s+s1}{Users}\PYG{l+s+s1}{\PYGZbs{}}\PYG{l+s+s1}{public}\PYG{l+s+s1}{\PYGZbs{}}\PYG{l+s+s1}{directory}\PYG{l+s+s1}{\PYGZsq{}}
\PYG{g+gp}{\PYGZgt{}\PYGZgt{}\PYGZgt{} }\PYG{n}{path\PYGZus{}build}\PYG{p}{(}\PYG{n}{path}\PYG{p}{,} \PYG{l+s+s1}{\PYGZsq{}}\PYG{l+s+s1}{..}\PYG{l+s+s1}{\PYGZbs{}}\PYG{l+s+s1}{other\PYGZus{}dir}\PYG{l+s+s1}{\PYGZsq{}}\PYG{p}{)}
\PYG{g+go}{\PYGZsq{}c:\PYGZbs{}Users\PYGZbs{}public\PYGZbs{}other\PYGZus{}dir\PYGZsq{}}
\end{sphinxVerbatim}

\end{description}\end{quote}

\end{fulllineitems}

\index{plain\_hex() (dans le module toolbox.tools)@\spxentry{plain\_hex()}\spxextra{dans le module toolbox.tools}}

\begin{fulllineitems}
\phantomsection\label{\detokenize{modules/tools:toolbox.tools.plain_hex}}\pysiglinewithargsret{\sphinxbfcode{\sphinxupquote{plain\_hex}}}{\emph{\DUrole{n}{hx}}, \emph{\DUrole{n}{s}\DUrole{o}{=}\DUrole{default_value}{3}}}{}
Complète un chiffre hexadecimal en préfixant une valeur de zéro
\begin{quote}\begin{description}
\item[{Paramètres}] \leavevmode\begin{itemize}
\item {} 
\sphinxstyleliteralstrong{\sphinxupquote{hx}} (\sphinxstyleliteralemphasis{\sphinxupquote{str}}) \textendash{} valeur hexadécimal

\item {} 
\sphinxstyleliteralstrong{\sphinxupquote{v}} (\sphinxstyleliteralemphasis{\sphinxupquote{int}}) \textendash{} longeur chaine attendu

\end{itemize}

\item[{Type renvoyé}] \leavevmode
str:

\item[{Examples}] \leavevmode
\begin{sphinxVerbatim}[commandchars=\\\{\}]
\PYG{g+gp}{\PYGZgt{}\PYGZgt{}\PYGZgt{} }\PYG{n}{hx} \PYG{o}{=} \PYG{l+s+s1}{\PYGZsq{}}\PYG{l+s+s1}{0x129}\PYG{l+s+s1}{\PYGZsq{}}
\PYG{g+gp}{\PYGZgt{}\PYGZgt{}\PYGZgt{} }\PYG{n}{plain\PYGZus{}hex}\PYG{p}{(}\PYG{n}{hx}\PYG{p}{,} \PYG{l+m+mi}{5}\PYG{p}{)}
\PYG{g+go}{0x00129}
\end{sphinxVerbatim}

\end{description}\end{quote}

\end{fulllineitems}

\index{plain\_zero() (dans le module toolbox.tools)@\spxentry{plain\_zero()}\spxextra{dans le module toolbox.tools}}

\begin{fulllineitems}
\phantomsection\label{\detokenize{modules/tools:toolbox.tools.plain_zero}}\pysiglinewithargsret{\sphinxbfcode{\sphinxupquote{plain\_zero}}}{\emph{\DUrole{n}{v}}, \emph{\DUrole{n}{s}}}{}
Complete une valeur chaine de zéro
\begin{quote}\begin{description}
\item[{Paramètres}] \leavevmode\begin{itemize}
\item {} 
\sphinxstyleliteralstrong{\sphinxupquote{v}} \textendash{} valeur à completer

\item {} 
\sphinxstyleliteralstrong{\sphinxupquote{s}} \textendash{} taille chaine attendu préfixé de zerom

\end{itemize}

\item[{Exemple}] \leavevmode
\begin{sphinxVerbatim}[commandchars=\\\{\}]
\PYG{g+gp}{\PYGZgt{}\PYGZgt{}\PYGZgt{} }\PYG{n}{d} \PYG{o}{=} \PYG{l+m+mi}{5}
\PYG{g+gp}{\PYGZgt{}\PYGZgt{}\PYGZgt{} }\PYG{n}{plain\PYGZus{}zero}\PYG{p}{(}\PYG{n}{d}\PYG{p}{,}\PYG{l+m+mi}{3}\PYG{p}{)}
\PYG{g+go}{\PYGZsq{}005\PYGZsq{}}
\end{sphinxVerbatim}

\end{description}\end{quote}

\end{fulllineitems}

\index{pop\_dic() (dans le module toolbox.tools)@\spxentry{pop\_dic()}\spxextra{dans le module toolbox.tools}}

\begin{fulllineitems}
\phantomsection\label{\detokenize{modules/tools:toolbox.tools.pop_dic}}\pysiglinewithargsret{\sphinxbfcode{\sphinxupquote{pop\_dic}}}{\emph{\DUrole{n}{l\_id}}, \emph{\DUrole{n}{dic}}}{}
Suppression d’une liste d’éléments d’un dictionnaire

:param list{[}str{]} l\_ids : liste de clé à supprimer
:param dict{[}str:object{]} dic: dictionaire à nettoyer

\end{fulllineitems}

\index{print\_err() (dans le module toolbox.tools)@\spxentry{print\_err()}\spxextra{dans le module toolbox.tools}}

\begin{fulllineitems}
\phantomsection\label{\detokenize{modules/tools:toolbox.tools.print_err}}\pysiglinewithargsret{\sphinxbfcode{\sphinxupquote{print\_err}}}{\emph{\DUrole{o}{*}\DUrole{n}{args}}, \emph{\DUrole{o}{**}\DUrole{n}{kwargs}}}{}
Ecriture sur le flux erreur de la console
\begin{quote}\begin{description}
\item[{Paramètres}] \leavevmode\begin{itemize}
\item {} 
\sphinxstyleliteralstrong{\sphinxupquote{args}} \textendash{} arguments 1

\item {} 
\sphinxstyleliteralstrong{\sphinxupquote{kwargs}} \textendash{} arguemnts2

\end{itemize}

\end{description}\end{quote}

\end{fulllineitems}

\index{pwd\_maker() (dans le module toolbox.tools)@\spxentry{pwd\_maker()}\spxextra{dans le module toolbox.tools}}

\begin{fulllineitems}
\phantomsection\label{\detokenize{modules/tools:toolbox.tools.pwd_maker}}\pysiglinewithargsret{\sphinxbfcode{\sphinxupquote{pwd\_maker}}}{\emph{\DUrole{n}{i\_size}\DUrole{o}{=}\DUrole{default_value}{8}}}{}
Génération d’un password respectant les regles de password
\begin{quote}\begin{description}
\item[{Conditions}] \leavevmode\begin{itemize}
\item {} 
Une majuscule

\item {} 
Une minuscule

\item {} 
Un chiffre

\item {} 
Un carectère spécial (@\#!?\$\&\sphinxhyphen{}\_ autorisé )

\end{itemize}

\item[{Paramètres}] \leavevmode
\sphinxstyleliteralstrong{\sphinxupquote{i\_size}} (\sphinxstyleliteralemphasis{\sphinxupquote{int}}) \textendash{} Nombre de caracteres de la chaine

\item[{Renvoie}] \leavevmode
Mot de passe

\end{description}\end{quote}

\end{fulllineitems}

\index{remove\_file() (dans le module toolbox.tools)@\spxentry{remove\_file()}\spxextra{dans le module toolbox.tools}}

\begin{fulllineitems}
\phantomsection\label{\detokenize{modules/tools:toolbox.tools.remove_file}}\pysiglinewithargsret{\sphinxbfcode{\sphinxupquote{remove\_file}}}{\emph{\DUrole{n}{p}}}{}
Suppression d’un fichier si existant
\begin{quote}\begin{description}
\item[{Paramètres}] \leavevmode
\sphinxstyleliteralstrong{\sphinxupquote{p}} (\sphinxstyleliteralemphasis{\sphinxupquote{str}}) \textendash{} chemin complet du fichier à supprimer

\end{description}\end{quote}

\end{fulllineitems}

\index{str\_dic() (dans le module toolbox.tools)@\spxentry{str\_dic()}\spxextra{dans le module toolbox.tools}}

\begin{fulllineitems}
\phantomsection\label{\detokenize{modules/tools:toolbox.tools.str_dic}}\pysiglinewithargsret{\sphinxbfcode{\sphinxupquote{str\_dic}}}{\emph{\DUrole{n}{chaine}}}{}
Convertion d’une chaine en distionnaire
\begin{quote}\begin{description}
\item[{Paramètres}] \leavevmode
\sphinxstyleliteralstrong{\sphinxupquote{chaine}} (\sphinxstyleliteralemphasis{\sphinxupquote{str}}) \textendash{} 

\item[{Type renvoyé}] \leavevmode
dic

\item[{Exemple}] \leavevmode
\begin{sphinxVerbatim}[commandchars=\\\{\}]
\PYG{g+gp}{\PYGZgt{}\PYGZgt{}\PYGZgt{} }\PYG{n}{s\PYGZus{}dic} \PYG{o}{=} \PYG{l+s+s2}{\PYGZdq{}}\PYG{l+s+s2}{\PYGZob{}}\PYG{l+s+s2}{\PYGZsq{}}\PYG{l+s+s2}{key}\PYG{l+s+s2}{\PYGZsq{}}\PYG{l+s+s2}{:value\PYGZcb{}}\PYG{l+s+s2}{\PYGZdq{}}
\PYG{g+gp}{\PYGZgt{}\PYGZgt{}\PYGZgt{} }\PYG{n}{str\PYGZus{}dic}\PYG{p}{(}\PYG{n}{s\PYGZus{}dic}\PYG{p}{)}
\PYG{g+go}{\PYGZob{}\PYGZsq{}key\PYGZsq{}: \PYGZsq{}value\PYGZsq{}\PYGZcb{}}
\end{sphinxVerbatim}

\end{description}\end{quote}

\end{fulllineitems}

\index{string\_me() (dans le module toolbox.tools)@\spxentry{string\_me()}\spxextra{dans le module toolbox.tools}}

\begin{fulllineitems}
\phantomsection\label{\detokenize{modules/tools:toolbox.tools.string_me}}\pysiglinewithargsret{\sphinxbfcode{\sphinxupquote{string\_me}}}{\emph{\DUrole{n}{v}}}{}
Convertion d’une valeur en chaine
\begin{quote}\begin{description}
\item[{Paramètres}] \leavevmode
\sphinxstyleliteralstrong{\sphinxupquote{v}} \textendash{} valeur à convertir

\item[{Type renvoyé}] \leavevmode
str, None en cas d’erreur

\end{description}\end{quote}

\end{fulllineitems}

\phantomsection\label{\detokenize{modules/tools:module-toolbox.features}}\index{module@\spxentry{module}!toolbox.features@\spxentry{toolbox.features}}\index{toolbox.features@\spxentry{toolbox.features}!module@\spxentry{module}}

\section{Module complémentaire}
\label{\detokenize{modules/tools:module-complementaire}}
pathfile : toolbox/features.py
\index{test\_http\_link() (dans le module toolbox.features)@\spxentry{test\_http\_link()}\spxextra{dans le module toolbox.features}}

\begin{fulllineitems}
\phantomsection\label{\detokenize{modules/tools:toolbox.features.test_http_link}}\pysiglinewithargsret{\sphinxbfcode{\sphinxupquote{test\_http\_link}}}{\emph{\DUrole{n}{url}}}{}
Vérifie une url et renvoie l’url valide
\begin{quote}\begin{description}
\item[{Paramètres}] \leavevmode
\sphinxstyleliteralstrong{\sphinxupquote{url}} \textendash{} url à évaluer

\item[{Rtype str}] \leavevmode
\end{description}\end{quote}

\end{fulllineitems}

\index{url\_join() (dans le module toolbox.features)@\spxentry{url\_join()}\spxextra{dans le module toolbox.features}}

\begin{fulllineitems}
\phantomsection\label{\detokenize{modules/tools:toolbox.features.url_join}}\pysiglinewithargsret{\sphinxbfcode{\sphinxupquote{url\_join}}}{\emph{\DUrole{n}{domaine}}, \emph{\DUrole{n}{page}}}{}
Generation d’une url

\end{fulllineitems}

\phantomsection\label{\detokenize{modules/dtemng:module-toolbox.dtemng}}\index{module@\spxentry{module}!toolbox.dtemng@\spxentry{toolbox.dtemng}}\index{toolbox.dtemng@\spxentry{toolbox.dtemng}!module@\spxentry{module}}

\chapter{Module de Gestion des date}
\label{\detokenize{modules/dtemng:module-de-gestion-des-date}}\label{\detokenize{modules/dtemng::doc}}
Liste de fonction pour utilisation des dates
pathfile : toolbox/tools


\section{Constantes globales}
\label{\detokenize{modules/dtemng:constantes-globales}}
Liste des jours de la semaine
I\_MON, I\_TUES, I\_WED,I\_THU, I\_FRI, I\_SAT, I\_SUN = 1, 2, 3, 4,5,6,7


\section{Fonctions}
\label{\detokenize{modules/dtemng:fonctions}}\index{date\_add\_workday() (dans le module toolbox.dtemng)@\spxentry{date\_add\_workday()}\spxextra{dans le module toolbox.dtemng}}

\begin{fulllineitems}
\phantomsection\label{\detokenize{modules/dtemng:toolbox.dtemng.date_add_workday}}\pysiglinewithargsret{\sphinxbfcode{\sphinxupquote{date\_add\_workday}}}{\emph{\DUrole{n}{dte}}, \emph{\DUrole{n}{nb}}}{}
Ajoute un nombre de jours ouvrés donnés à une date
\begin{quote}\begin{description}
\item[{Param}] \leavevmode
datetime dte: date de référence

\item[{Paramètres}] \leavevmode
\sphinxstyleliteralstrong{\sphinxupquote{nb}} (\sphinxstyleliteralemphasis{\sphinxupquote{int}}) \textendash{} nombre de jour à additionner (valeur négative/positive)

\item[{Renvoie}] \leavevmode
date de depat + nombre de jours

\end{description}\end{quote}

\end{fulllineitems}

\index{date\_dayed() (dans le module toolbox.dtemng)@\spxentry{date\_dayed()}\spxextra{dans le module toolbox.dtemng}}

\begin{fulllineitems}
\phantomsection\label{\detokenize{modules/dtemng:toolbox.dtemng.date_dayed}}\pysiglinewithargsret{\sphinxbfcode{\sphinxupquote{date\_dayed}}}{\emph{\DUrole{n}{dte}\DUrole{o}{=}\DUrole{default_value}{None}}, \emph{\DUrole{n}{b}\DUrole{o}{=}\DUrole{default_value}{True}}}{}
Positionne la date indiquée à minuit au matin ou au soir
\begin{quote}\begin{description}
\item[{Paramètres}] \leavevmode\begin{itemize}
\item {} 
\sphinxstyleliteralstrong{\sphinxupquote{dte}} (\sphinxstyleliteralemphasis{\sphinxupquote{datetime}}) \textendash{} Date

\item {} 
\sphinxstyleliteralstrong{\sphinxupquote{b}} (\sphinxstyleliteralemphasis{\sphinxupquote{bool}}) \textendash{} Date debut de jour (00:00:00.000) ou date de fin de journee date du jour + 1 (minuit) soit lendemin à 00

\end{itemize}

\item[{Example}] \leavevmode
\begin{sphinxVerbatim}[commandchars=\\\{\}]
\PYG{g+gp}{\PYGZgt{}\PYGZgt{}\PYGZgt{} }\PYG{n}{date\PYGZus{}dayed}\PYG{p}{(}\PYG{p}{)}
\PYG{g+go}{datetime.datetime(2020, 12, 19, 0, 0,datetime.datetime(2020, 12, 19, 0, 0)}
\end{sphinxVerbatim}

\end{description}\end{quote}

\end{fulllineitems}

\index{date\_rss() (dans le module toolbox.dtemng)@\spxentry{date\_rss()}\spxextra{dans le module toolbox.dtemng}}

\begin{fulllineitems}
\phantomsection\label{\detokenize{modules/dtemng:toolbox.dtemng.date_rss}}\pysiglinewithargsret{\sphinxbfcode{\sphinxupquote{date\_rss}}}{\emph{\DUrole{n}{dte}\DUrole{o}{=}\DUrole{default_value}{None}}}{}
Dtate au format RSS

\end{fulllineitems}

\index{dateadd() (dans le module toolbox.dtemng)@\spxentry{dateadd()}\spxextra{dans le module toolbox.dtemng}}

\begin{fulllineitems}
\phantomsection\label{\detokenize{modules/dtemng:toolbox.dtemng.dateadd}}\pysiglinewithargsret{\sphinxbfcode{\sphinxupquote{dateadd}}}{\emph{\DUrole{n}{dte}}, \emph{\DUrole{n}{nb}}, \emph{\DUrole{n}{fm}\DUrole{o}{=}\DUrole{default_value}{\textquotesingle{}d\textquotesingle{}}}}{}
Ajoute un nombre de jours données à une date
\begin{quote}\begin{description}
\item[{Paramètres}] \leavevmode\begin{itemize}
\item {} 
\sphinxstyleliteralstrong{\sphinxupquote{dte}} (\sphinxstyleliteralemphasis{\sphinxupquote{date}}) \textendash{} date de départ

\item {} 
\sphinxstyleliteralstrong{\sphinxupquote{nb}} (\sphinxstyleliteralemphasis{\sphinxupquote{int}}) \textendash{} nombre de jour à additionner (valeur négative/positive)

\item {} 
\sphinxstyleliteralstrong{\sphinxupquote{fm}} (\sphinxstyleliteralemphasis{\sphinxupquote{str}}) \textendash{} \begin{itemize}
\item {} 
days (default), * h (hours), * m (minutes)

\end{itemize}


\end{itemize}

\item[{Type renvoyé}] \leavevmode
datetime

\end{description}\end{quote}

\end{fulllineitems}

\index{datepaques() (dans le module toolbox.dtemng)@\spxentry{datepaques()}\spxextra{dans le module toolbox.dtemng}}

\begin{fulllineitems}
\phantomsection\label{\detokenize{modules/dtemng:toolbox.dtemng.datepaques}}\pysiglinewithargsret{\sphinxbfcode{\sphinxupquote{datepaques}}}{\emph{\DUrole{n}{y}}}{}
Dates Pâques d’une année donnée
\begin{itemize}
\item {} 
Lundi de paque : lundi suivant le dimanche de paque (La Pâque)

\item {} 
Jeudi de l’ascension : 3 jour après paques

\item {} 
pentecote : 49 jours après le lundi de paques

\end{itemize}
\begin{quote}\begin{description}
\item[{Paramètres}] \leavevmode
\sphinxstyleliteralstrong{\sphinxupquote{y}} (\sphinxstyleliteralemphasis{\sphinxupquote{int}}) \textendash{} année de référence

\item[{Type renvoyé}] \leavevmode
list{[}date{]}

\end{description}\end{quote}

\end{fulllineitems}

\index{datestr() (dans le module toolbox.dtemng)@\spxentry{datestr()}\spxextra{dans le module toolbox.dtemng}}

\begin{fulllineitems}
\phantomsection\label{\detokenize{modules/dtemng:toolbox.dtemng.datestr}}\pysiglinewithargsret{\sphinxbfcode{\sphinxupquote{datestr}}}{\emph{\DUrole{n}{dte}\DUrole{o}{=}\DUrole{default_value}{None}}, \emph{\DUrole{n}{fm}\DUrole{o}{=}\DUrole{default_value}{\textquotesingle{}\%Y\sphinxhyphen{}\%m\sphinxhyphen{}\%dT\%H:\%M:\%S\textquotesingle{}}}}{}
Convertit une date en chaine selon un format donnée
\begin{quote}\begin{description}
\item[{Paramètres}] \leavevmode\begin{itemize}
\item {} 
\sphinxstyleliteralstrong{\sphinxupquote{dte}} (\sphinxstyleliteralemphasis{\sphinxupquote{datetime}}) \textendash{} date à convertir date du jour par défaut

\item {} 
\sphinxstyleliteralstrong{\sphinxupquote{fm}} (\sphinxstyleliteralemphasis{\sphinxupquote{str}}) \textendash{} format désirée, defaults to  “\%d/\%m/\%Y”

\end{itemize}

\item[{Renvoie}] \leavevmode
Renvoie un chaine correspondant au format date passé en parametre

\item[{Type renvoyé}] \leavevmode
str

\item[{Exemple}] \leavevmode
\begin{sphinxVerbatim}[commandchars=\\\{\}]
\PYG{g+gp}{\PYGZgt{}\PYGZgt{}\PYGZgt{} }\PYG{n}{d} \PYG{o}{=} \PYG{n}{maintenant} \PYG{p}{(}\PYG{p}{)}
\PYG{g+gp}{\PYGZgt{}\PYGZgt{}\PYGZgt{} }\PYG{n}{datestr} \PYG{p}{(}\PYG{n}{d}\PYG{p}{,} \PYG{l+s+s1}{\PYGZsq{}}\PYG{l+s+si}{\PYGZpc{}d}\PYG{l+s+s1}{.}\PYG{l+s+s1}{\PYGZpc{}}\PYG{l+s+s1}{m.}\PYG{l+s+s1}{\PYGZpc{}}\PYG{l+s+s1}{Y}\PYG{l+s+s1}{\PYGZsq{}}\PYG{p}{)}
\PYG{g+go}{02.06.2019}
\end{sphinxVerbatim}

\end{description}\end{quote}

\end{fulllineitems}

\index{datetime\_from\_local\_to\_utc() (dans le module toolbox.dtemng)@\spxentry{datetime\_from\_local\_to\_utc()}\spxextra{dans le module toolbox.dtemng}}

\begin{fulllineitems}
\phantomsection\label{\detokenize{modules/dtemng:toolbox.dtemng.datetime_from_local_to_utc}}\pysiglinewithargsret{\sphinxbfcode{\sphinxupquote{datetime\_from\_local\_to\_utc}}}{\emph{\DUrole{n}{utc\_datetime}}}{}
Convertie une date et heure local en heure utc
\begin{quote}\begin{description}
\item[{Paramètres}] \leavevmode
\sphinxstyleliteralstrong{\sphinxupquote{utc\_datetime}} (\sphinxstyleliteralemphasis{\sphinxupquote{date}}) \textendash{} datetime local

\item[{Renvoie}] \leavevmode
datatime utc

\end{description}\end{quote}

\end{fulllineitems}

\index{datetime\_from\_utc\_to\_local() (dans le module toolbox.dtemng)@\spxentry{datetime\_from\_utc\_to\_local()}\spxextra{dans le module toolbox.dtemng}}

\begin{fulllineitems}
\phantomsection\label{\detokenize{modules/dtemng:toolbox.dtemng.datetime_from_utc_to_local}}\pysiglinewithargsret{\sphinxbfcode{\sphinxupquote{datetime\_from\_utc\_to\_local}}}{\emph{\DUrole{n}{utc\_datetime}}}{}
Convertie la date et heure donné (utc) en date local
\begin{quote}\begin{description}
\item[{Paramètres}] \leavevmode
\sphinxstyleliteralstrong{\sphinxupquote{utc\_datetime}} \textendash{} datetime utc

\item[{Renvoie}] \leavevmode
date locale

\end{description}\end{quote}

\end{fulllineitems}

\index{day\_in\_hour() (dans le module toolbox.dtemng)@\spxentry{day\_in\_hour()}\spxextra{dans le module toolbox.dtemng}}

\begin{fulllineitems}
\phantomsection\label{\detokenize{modules/dtemng:toolbox.dtemng.day_in_hour}}\pysiglinewithargsret{\sphinxbfcode{\sphinxupquote{day\_in\_hour}}}{\emph{\DUrole{n}{dy}}}{}
Convertion d’un nombre de jours en heure
\begin{quote}\begin{description}
\item[{Paramètres}] \leavevmode
\sphinxstyleliteralstrong{\sphinxupquote{dy}} (\sphinxstyleliteralemphasis{\sphinxupquote{int}}) \textendash{} nombre de jours

\item[{Type renvoyé}] \leavevmode
int

\end{description}\end{quote}

\end{fulllineitems}

\index{day\_in\_sec() (dans le module toolbox.dtemng)@\spxentry{day\_in\_sec()}\spxextra{dans le module toolbox.dtemng}}

\begin{fulllineitems}
\phantomsection\label{\detokenize{modules/dtemng:toolbox.dtemng.day_in_sec}}\pysiglinewithargsret{\sphinxbfcode{\sphinxupquote{day\_in\_sec}}}{\emph{\DUrole{n}{dy}}, \emph{\DUrole{n}{ml}\DUrole{o}{=}\DUrole{default_value}{False}}}{}
Convertion d’un nombre de jours en secondes ou milisecondes
\begin{quote}\begin{description}
\item[{Paramètres}] \leavevmode\begin{itemize}
\item {} 
\sphinxstyleliteralstrong{\sphinxupquote{dy}} (\sphinxstyleliteralemphasis{\sphinxupquote{int}}) \textendash{} nombre de jours

\item {} 
\sphinxstyleliteralstrong{\sphinxupquote{ml}} (\sphinxstyleliteralemphasis{\sphinxupquote{bool}}) \textendash{} en millisecondes si True sinon en secondes, dafault False

\end{itemize}

\item[{Renvoie}] \leavevmode
(milli) secondes

\end{description}\end{quote}

\end{fulllineitems}

\index{dtediff() (dans le module toolbox.dtemng)@\spxentry{dtediff()}\spxextra{dans le module toolbox.dtemng}}

\begin{fulllineitems}
\phantomsection\label{\detokenize{modules/dtemng:toolbox.dtemng.dtediff}}\pysiglinewithargsret{\sphinxbfcode{\sphinxupquote{dtediff}}}{\emph{\DUrole{n}{dtea}}, \emph{\DUrole{n}{dteb}}}{}
Calcul du nombre de jours entre deux dates
\begin{quote}\begin{description}
\item[{Paramètres}] \leavevmode\begin{itemize}
\item {} 
\sphinxstyleliteralstrong{\sphinxupquote{datea}} (\sphinxstyleliteralemphasis{\sphinxupquote{datettime}}) \textendash{} date à comparer

\item {} 
\sphinxstyleliteralstrong{\sphinxupquote{dateb}} (\sphinxstyleliteralemphasis{\sphinxupquote{datetime}}) \textendash{} date à comparer

\end{itemize}

\item[{Type renvoyé}] \leavevmode
int

\end{description}\end{quote}

\end{fulllineitems}

\index{dtets() (dans le module toolbox.dtemng)@\spxentry{dtets()}\spxextra{dans le module toolbox.dtemng}}

\begin{fulllineitems}
\phantomsection\label{\detokenize{modules/dtemng:toolbox.dtemng.dtets}}\pysiglinewithargsret{\sphinxbfcode{\sphinxupquote{dtets}}}{\emph{\DUrole{n}{dte}\DUrole{o}{=}\DUrole{default_value}{None}}}{}
Conversion date \sphinxhyphen{} timestamp
\begin{quote}\begin{description}
\item[{Paramètres}] \leavevmode
\sphinxstyleliteralstrong{\sphinxupquote{dt}} (\sphinxstyleliteralemphasis{\sphinxupquote{date}}) \textendash{} date à convertir

\item[{Return int}] \leavevmode
date en miliseconde (sans les ms)

\end{description}\end{quote}

\end{fulllineitems}

\index{fullmonth() (dans le module toolbox.dtemng)@\spxentry{fullmonth()}\spxextra{dans le module toolbox.dtemng}}

\begin{fulllineitems}
\phantomsection\label{\detokenize{modules/dtemng:toolbox.dtemng.fullmonth}}\pysiglinewithargsret{\sphinxbfcode{\sphinxupquote{fullmonth}}}{\emph{\DUrole{n}{dte}}}{}
Renvoie la date du jour au format MOIS YYYY
\begin{quote}\begin{description}
\item[{Paramètres}] \leavevmode
\sphinxstyleliteralstrong{\sphinxupquote{dte}} (\sphinxstyleliteralemphasis{\sphinxupquote{datetime}}) \textendash{} 

\item[{Type renvoyé}] \leavevmode
str

\end{description}\end{quote}

\end{fulllineitems}

\index{get\_date() (dans le module toolbox.dtemng)@\spxentry{get\_date()}\spxextra{dans le module toolbox.dtemng}}

\begin{fulllineitems}
\phantomsection\label{\detokenize{modules/dtemng:toolbox.dtemng.get_date}}\pysiglinewithargsret{\sphinxbfcode{\sphinxupquote{get\_date}}}{\emph{\DUrole{n}{p\_year}}, \emph{\DUrole{n}{p\_month}}, \emph{\DUrole{n}{p\_day}}}{}
Generationd’une date a partir des valeur numerique
\begin{quote}\begin{description}
\item[{Paramètres}] \leavevmode\begin{itemize}
\item {} 
\sphinxstyleliteralstrong{\sphinxupquote{p\_year}} (\sphinxstyleliteralemphasis{\sphinxupquote{int}}) \textendash{} année

\item {} 
\sphinxstyleliteralstrong{\sphinxupquote{p\_month}} (\sphinxstyleliteralemphasis{\sphinxupquote{int}}) \textendash{} mois

\item {} 
\sphinxstyleliteralstrong{\sphinxupquote{p\_day}} (\sphinxstyleliteralemphasis{\sphinxupquote{int}}) \textendash{} date jour

\end{itemize}

\item[{Type renvoyé}] \leavevmode
datetime

\end{description}\end{quote}

\end{fulllineitems}

\index{get\_time() (dans le module toolbox.dtemng)@\spxentry{get\_time()}\spxextra{dans le module toolbox.dtemng}}

\begin{fulllineitems}
\phantomsection\label{\detokenize{modules/dtemng:toolbox.dtemng.get_time}}\pysiglinewithargsret{\sphinxbfcode{\sphinxupquote{get\_time}}}{\emph{\DUrole{n}{dte}}}{}
Renvoie d’une date au format time
:param datetime dte:
:rtype: time

\end{fulllineitems}

\index{get\_weeks\_num() (dans le module toolbox.dtemng)@\spxentry{get\_weeks\_num()}\spxextra{dans le module toolbox.dtemng}}

\begin{fulllineitems}
\phantomsection\label{\detokenize{modules/dtemng:toolbox.dtemng.get_weeks_num}}\pysiglinewithargsret{\sphinxbfcode{\sphinxupquote{get\_weeks\_num}}}{\emph{\DUrole{n}{dte}\DUrole{o}{=}\DUrole{default_value}{None}}}{}
Renvoie le numéro de la date indiqué (now par deafut)

\end{fulllineitems}

\index{is\_workday() (dans le module toolbox.dtemng)@\spxentry{is\_workday()}\spxextra{dans le module toolbox.dtemng}}

\begin{fulllineitems}
\phantomsection\label{\detokenize{modules/dtemng:toolbox.dtemng.is_workday}}\pysiglinewithargsret{\sphinxbfcode{\sphinxupquote{is\_workday}}}{\emph{\DUrole{n}{dte}}}{}
Determine si la date est un jour ouvré ou vaqué (week\sphinxhyphen{}end / fériés)
\begin{quote}\begin{description}
\item[{Paramètres}] \leavevmode
\sphinxstyleliteralstrong{\sphinxupquote{dte}} \textendash{} date à évaluer

\item[{Renvoie}] \leavevmode
renvoie le statut jour ouvré (true=ouvré)

\item[{Rtype bool}] \leavevmode
\end{description}\end{quote}

\end{fulllineitems}

\index{isotodate() (dans le module toolbox.dtemng)@\spxentry{isotodate()}\spxextra{dans le module toolbox.dtemng}}

\begin{fulllineitems}
\phantomsection\label{\detokenize{modules/dtemng:toolbox.dtemng.isotodate}}\pysiglinewithargsret{\sphinxbfcode{\sphinxupquote{isotodate}}}{\emph{\DUrole{n}{s\_iso}}}{}
Conversion str\_iso \sphinxhyphen{} date

Format ISO : YYYY\sphinxhyphen{}MM\sphinxhyphen{}DDTHH:MN
:param p\_dte:
\begin{quote}\begin{description}
\item[{Renvoie}] \leavevmode


\end{description}\end{quote}

\end{fulllineitems}

\index{jours\_feries() (dans le module toolbox.dtemng)@\spxentry{jours\_feries()}\spxextra{dans le module toolbox.dtemng}}

\begin{fulllineitems}
\phantomsection\label{\detokenize{modules/dtemng:toolbox.dtemng.jours_feries}}\pysiglinewithargsret{\sphinxbfcode{\sphinxupquote{jours\_feries}}}{\emph{\DUrole{n}{y}\DUrole{o}{=}\DUrole{default_value}{None}}}{}
Jour fériés pour une date donnée
\begin{quote}\begin{description}
\item[{Paramètres}] \leavevmode
\sphinxstyleliteralstrong{\sphinxupquote{y}} (\sphinxstyleliteralemphasis{\sphinxupquote{int}}) \textendash{} Année de référence (optionnel), default : année en cours

\item[{Renvoie}] \leavevmode
un tableau de date de jours fériés

\item[{Exemple}] \leavevmode
\begin{sphinxVerbatim}[commandchars=\\\{\}]
\PYG{g+gp}{\PYGZgt{}\PYGZgt{}\PYGZgt{} }\PYG{n}{jours\PYGZus{}feries} \PYG{p}{(}\PYG{p}{)}             \PYG{c+c1}{\PYGZsh{}Jours fériés année en cours}
\PYG{g+gp}{\PYGZgt{}\PYGZgt{}\PYGZgt{} }\PYG{n}{jours\PYGZus{}feries} \PYG{p}{(}\PYG{l+m+mi}{2018}\PYG{p}{)} \PYG{c+c1}{\PYGZsh{} jours fériés année 2018}
\end{sphinxVerbatim}

\end{description}\end{quote}

\end{fulllineitems}

\index{maintenant() (dans le module toolbox.dtemng)@\spxentry{maintenant()}\spxextra{dans le module toolbox.dtemng}}

\begin{fulllineitems}
\phantomsection\label{\detokenize{modules/dtemng:toolbox.dtemng.maintenant}}\pysiglinewithargsret{\sphinxbfcode{\sphinxupquote{maintenant}}}{\emph{\DUrole{n}{utc}\DUrole{o}{=}\DUrole{default_value}{False}}, \emph{\DUrole{n}{fm}\DUrole{o}{=}\DUrole{default_value}{None}}, \emph{\DUrole{n}{tz}\DUrole{o}{=}\DUrole{default_value}{None}}}{}
Date et heure de l’instant (Now)
\begin{quote}\begin{description}
\item[{Paramètres}] \leavevmode\begin{itemize}
\item {} 
\sphinxstyleliteralstrong{\sphinxupquote{utc}} (\sphinxstyleliteralemphasis{\sphinxupquote{bool}}) \textendash{} Si True renvoie de l’heure UTC (GMT) ou l’heure local

\item {} 
\sphinxstyleliteralstrong{\sphinxupquote{p\_iso}} (\sphinxstyleliteralemphasis{\sphinxupquote{bool}}) \textendash{} “iso” = Format iso | “ts” = timeimestamp | None = datetime

\end{itemize}

\item[{Type renvoyé}] \leavevmode
datetime | string

\item[{Exemple}] \leavevmode
\begin{sphinxVerbatim}[commandchars=\\\{\}]
\PYG{g+gp}{\PYGZgt{}\PYGZgt{}\PYGZgt{} }\PYG{n}{maintenant} \PYG{p}{(}\PYG{p}{)}
\PYG{g+go}{datetime.datetime (2019, 06, 02, 17, 30, 43, 248622)}
\PYG{g+gp}{\PYGZgt{}\PYGZgt{}\PYGZgt{} }\PYG{n}{maintenant} \PYG{p}{(}\PYG{k+kc}{True}\PYG{p}{)}
\PYG{g+go}{\PYGZsq{}2019\PYGZhy{}06\PYGZhy{}02T17:30:43.248622\PYGZsq{}}
\end{sphinxVerbatim}

\end{description}\end{quote}

\end{fulllineitems}

\index{set\_timezone() (dans le module toolbox.dtemng)@\spxentry{set\_timezone()}\spxextra{dans le module toolbox.dtemng}}

\begin{fulllineitems}
\phantomsection\label{\detokenize{modules/dtemng:toolbox.dtemng.set_timezone}}\pysiglinewithargsret{\sphinxbfcode{\sphinxupquote{set\_timezone}}}{\emph{dt}, \emph{tz=\textless{}UTC\textgreater{}}}{}
Applique la timezone indiquée à la date passée en parametre
\begin{quote}\begin{description}
\item[{Paramètres}] \leavevmode\begin{itemize}
\item {} 
\sphinxstyleliteralstrong{\sphinxupquote{dt}} (\sphinxstyleliteralemphasis{\sphinxupquote{date}}) \textendash{} date

\item {} 
\sphinxstyleliteralstrong{\sphinxupquote{tz}} (\sphinxstyleliteralemphasis{\sphinxupquote{timezone}}) \textendash{} timezone

\end{itemize}

\end{description}\end{quote}

\end{fulllineitems}

\index{strdate() (dans le module toolbox.dtemng)@\spxentry{strdate()}\spxextra{dans le module toolbox.dtemng}}

\begin{fulllineitems}
\phantomsection\label{\detokenize{modules/dtemng:toolbox.dtemng.strdate}}\pysiglinewithargsret{\sphinxbfcode{\sphinxupquote{strdate}}}{\emph{\DUrole{n}{dt}}, \emph{\DUrole{n}{pt}\DUrole{o}{=}\DUrole{default_value}{\textquotesingle{}\%d\sphinxhyphen{}\%m\sphinxhyphen{}\%Y \%H:\%M:\%S\textquotesingle{}}}}{}
Conversion string \sphinxhyphen{} date
\begin{quote}\begin{description}
\item[{Paramètres}] \leavevmode\begin{itemize}
\item {} 
\sphinxstyleliteralstrong{\sphinxupquote{dt}} (\sphinxstyleliteralemphasis{\sphinxupquote{str}}) \textendash{} date

\item {} 
\sphinxstyleliteralstrong{\sphinxupquote{pt}}\sphinxstyleliteralstrong{\sphinxupquote{, }}\sphinxstyleliteralstrong{\sphinxupquote{default \textquotesingle{}\%d\sphinxhyphen{}\%m\sphinxhyphen{}\%Y \%H:\%M:\%S\textquotesingle{}}} (\sphinxstyleliteralemphasis{\sphinxupquote{str}}) \textendash{} patterne, optional

\end{itemize}

\item[{Renvoie}] \leavevmode
Renvoie la date convertit ou None en cas d’invalidité (date non conform)

\item[{Type renvoyé}] \leavevmode
datetime

\item[{Exemple}] \leavevmode
\begin{sphinxVerbatim}[commandchars=\\\{\}]
\PYG{g+gp}{\PYGZgt{}\PYGZgt{}\PYGZgt{} }\PYG{n}{s} \PYG{o}{=} \PYG{l+s+s1}{\PYGZsq{}}\PYG{l+s+s1}{24\PYGZhy{}O2\PYGZhy{}1976 16:45}\PYG{l+s+s1}{\PYGZsq{}}
\PYG{g+gp}{\PYGZgt{}\PYGZgt{}\PYGZgt{} }\PYG{n}{strdate} \PYG{p}{(}\PYG{n}{s}\PYG{p}{,} \PYG{l+s+s1}{\PYGZsq{}}\PYG{l+s+si}{\PYGZpc{}d}\PYG{l+s+s1}{\PYGZhy{}}\PYG{l+s+s1}{\PYGZpc{}}\PYG{l+s+s1}{m\PYGZhy{}}\PYG{l+s+s1}{\PYGZpc{}}\PYG{l+s+s1}{Y }\PYG{l+s+s1}{\PYGZpc{}}\PYG{l+s+s1}{H:}\PYG{l+s+s1}{\PYGZpc{}}\PYG{l+s+s1}{m}\PYG{l+s+s1}{\PYGZsq{}}\PYG{p}{)}
\PYG{g+go}{datetime.datetime(1976,02,24,16,45)}
\end{sphinxVerbatim}

\end{description}\end{quote}

\end{fulllineitems}

\index{timeadd() (dans le module toolbox.dtemng)@\spxentry{timeadd()}\spxextra{dans le module toolbox.dtemng}}

\begin{fulllineitems}
\phantomsection\label{\detokenize{modules/dtemng:toolbox.dtemng.timeadd}}\pysiglinewithargsret{\sphinxbfcode{\sphinxupquote{timeadd}}}{\emph{\DUrole{n}{dte}}, \emph{\DUrole{n}{nb}}}{}
Ajoute un nombre d’heure données à une date
\begin{quote}\begin{description}
\item[{Paramètres}] \leavevmode\begin{itemize}
\item {} 
\sphinxstyleliteralstrong{\sphinxupquote{dte}} (\sphinxstyleliteralemphasis{\sphinxupquote{date}}) \textendash{} date de départ

\item {} 
\sphinxstyleliteralstrong{\sphinxupquote{nb}} (\sphinxstyleliteralemphasis{\sphinxupquote{int}}) \textendash{} nombre de jour à additionner (valeur négative/positive)

\end{itemize}

\item[{Renvoie}] \leavevmode
date de depat + nombre de jours

\end{description}\end{quote}

\end{fulllineitems}

\index{today() (dans le module toolbox.dtemng)@\spxentry{today()}\spxextra{dans le module toolbox.dtemng}}

\begin{fulllineitems}
\phantomsection\label{\detokenize{modules/dtemng:toolbox.dtemng.today}}\pysiglinewithargsret{\sphinxbfcode{\sphinxupquote{today}}}{\emph{\DUrole{n}{fm}\DUrole{o}{=}\DUrole{default_value}{\textquotesingle{}\%d/\%m/\%Y\textquotesingle{}}}}{}
Renvoie la date du jour
\begin{quote}\begin{description}
\item[{Paramètres}] \leavevmode
\sphinxstyleliteralstrong{\sphinxupquote{fm}} (\sphinxstyleliteralemphasis{\sphinxupquote{str}}) \textendash{} Format de la date attendu

\item[{Type renvoyé}] \leavevmode
str

\item[{Exemple}] \leavevmode
\begin{sphinxVerbatim}[commandchars=\\\{\}]
\PYG{g+gp}{\PYGZgt{}\PYGZgt{}\PYGZgt{} }\PYG{n}{today} \PYG{p}{(}\PYG{p}{)}
\PYG{g+go}{\PYGZsq{}02/06/2019\PYGZsq{}}
\PYG{g+gp}{\PYGZgt{}\PYGZgt{}\PYGZgt{} }\PYG{n}{today}\PYG{p}{(}\PYG{l+s+s1}{\PYGZsq{}}\PYG{l+s+si}{\PYGZpc{}d}\PYG{l+s+s1}{.}\PYG{l+s+s1}{\PYGZpc{}}\PYG{l+s+s1}{m.}\PYG{l+s+s1}{\PYGZpc{}}\PYG{l+s+s1}{Y}\PYG{l+s+s1}{\PYGZsq{}}\PYG{p}{)}
\PYG{g+go}{\PYGZsq{}02.06.2019\PYGZsq{}}
\end{sphinxVerbatim}

\end{description}\end{quote}

\end{fulllineitems}

\index{tsdate() (dans le module toolbox.dtemng)@\spxentry{tsdate()}\spxextra{dans le module toolbox.dtemng}}

\begin{fulllineitems}
\phantomsection\label{\detokenize{modules/dtemng:toolbox.dtemng.tsdate}}\pysiglinewithargsret{\sphinxbfcode{\sphinxupquote{tsdate}}}{\emph{\DUrole{n}{ts}}}{}
Conversion timestamp \sphinxhyphen{} date
\begin{quote}\begin{description}
\item[{Paramètres}] \leavevmode
\sphinxstyleliteralstrong{\sphinxupquote{ts}} \textendash{} temps en milliseconde depuis 1970

\item[{Renvoie}] \leavevmode
date

\end{description}\end{quote}

\end{fulllineitems}

\index{tstring() (dans le module toolbox.dtemng)@\spxentry{tstring()}\spxextra{dans le module toolbox.dtemng}}

\begin{fulllineitems}
\phantomsection\label{\detokenize{modules/dtemng:toolbox.dtemng.tstring}}\pysiglinewithargsret{\sphinxbfcode{\sphinxupquote{tstring}}}{\emph{\DUrole{n}{ts}}, \emph{\DUrole{n}{fm}\DUrole{o}{=}\DUrole{default_value}{\textquotesingle{}\%Y.\%m.\%d\sphinxhyphen{}\%H:\%M (\%a)\textquotesingle{}}}}{}
Conversion timestamp \sphinxhyphen{} chaine(str)
\begin{quote}\begin{description}
\item[{Paramètres}] \leavevmode\begin{itemize}
\item {} 
\sphinxstyleliteralstrong{\sphinxupquote{ts}} (\sphinxstyleliteralemphasis{\sphinxupquote{int}}) \textendash{} timestamp

\item {} 
\sphinxstyleliteralstrong{\sphinxupquote{fm}} (\sphinxstyleliteralemphasis{\sphinxupquote{str}}) \textendash{} format attendu

\end{itemize}

\item[{Type renvoyé}] \leavevmode
str

\end{description}\end{quote}

\end{fulllineitems}

\index{utcnow\_iso() (dans le module toolbox.dtemng)@\spxentry{utcnow\_iso()}\spxextra{dans le module toolbox.dtemng}}

\begin{fulllineitems}
\phantomsection\label{\detokenize{modules/dtemng:toolbox.dtemng.utcnow_iso}}\pysiglinewithargsret{\sphinxbfcode{\sphinxupquote{utcnow\_iso}}}{}{}
Date et heure actuelle utc au format iso
:return: date utc

\end{fulllineitems}

\index{utcnow\_ts() (dans le module toolbox.dtemng)@\spxentry{utcnow\_ts()}\spxextra{dans le module toolbox.dtemng}}

\begin{fulllineitems}
\phantomsection\label{\detokenize{modules/dtemng:toolbox.dtemng.utcnow_ts}}\pysiglinewithargsret{\sphinxbfcode{\sphinxupquote{utcnow\_ts}}}{}{}
Date et heure actuelle utc au format timestamp
:return: timestamp

\end{fulllineitems}



\chapter{Gestion configuration}
\label{\detokenize{classes/cfgloader:module-toolbox.cfgmng}}\label{\detokenize{classes/cfgloader:gestion-configuration}}\label{\detokenize{classes/cfgloader::doc}}\index{module@\spxentry{module}!toolbox.cfgmng@\spxentry{toolbox.cfgmng}}\index{toolbox.cfgmng@\spxentry{toolbox.cfgmng}!module@\spxentry{module}}
Gestion fichiers de configurations (YAML)

pathfile : toolbox/cfgmng.py


\section{Repertoires par défaut}
\label{\detokenize{classes/cfgloader:repertoires-par-defaut}}
\begin{sphinxadmonition}{note}{Note:}\begin{itemize}
\item {} 
PROJECT\_DIR/cfg/PROJECT\_DIR/cfg/.log.yml : Fichier de configuration des logs

\item {} 
PROJECT\_DIR/cfg/.app.yml : Fichier de configuration de l’application

\item {} 
PROJECT\_DIR/cfg/categorie.yml : Fichier de liste définie par un code et un libelle

\item {} 
PROJECT\_DIR/cfg/mailing.yml : Fichier de mails préparés

\item {} 
PROJECT\_DIR/cfg/validators.yml : Fichier de validation(cf CERBERUS)

\item {} 
PROJECT\_DIR/cfg/normalizor.yml : Fichier de normalization(cf CERBERUS)

\end{itemize}
\end{sphinxadmonition}


\section{Class CFBases}
\label{\detokenize{classes/cfgloader:class-cfbases}}\index{CFGBases (classe dans toolbox.cfgmng)@\spxentry{CFGBases}\spxextra{classe dans toolbox.cfgmng}}

\begin{fulllineitems}
\phantomsection\label{\detokenize{classes/cfgloader:toolbox.cfgmng.CFGBases}}\pysigline{\sphinxbfcode{\sphinxupquote{class }}\sphinxbfcode{\sphinxupquote{CFGBases}}}
Cette class permet de gere des fichiers de configuration disponibles dans le repertoire \textless{}PROJET\_DIR\textgreater{}/cfg
\index{app\_cfg() (méthode statique CFGBases)@\spxentry{app\_cfg()}\spxextra{méthode statique CFGBases}}

\begin{fulllineitems}
\phantomsection\label{\detokenize{classes/cfgloader:toolbox.cfgmng.CFGBases.app_cfg}}\pysiglinewithargsret{\sphinxbfcode{\sphinxupquote{static }}\sphinxbfcode{\sphinxupquote{app\_cfg}}}{\emph{\DUrole{n}{code}\DUrole{o}{=}\DUrole{default_value}{None}}}{}
Parametres application
\begin{quote}\begin{description}
\item[{Paramètres}] \leavevmode
\sphinxstyleliteralstrong{\sphinxupquote{code}} (\sphinxstyleliteralemphasis{\sphinxupquote{str}}) \textendash{} clé a retourner (filtre)

\item[{Renvoie}] \leavevmode
Configuration

\end{description}\end{quote}

\end{fulllineitems}

\index{categorie\_lib() (méthode statique CFGBases)@\spxentry{categorie\_lib()}\spxextra{méthode statique CFGBases}}

\begin{fulllineitems}
\phantomsection\label{\detokenize{classes/cfgloader:toolbox.cfgmng.CFGBases.categorie_lib}}\pysiglinewithargsret{\sphinxbfcode{\sphinxupquote{static }}\sphinxbfcode{\sphinxupquote{categorie\_lib}}}{\emph{\DUrole{n}{code}\DUrole{o}{=}\DUrole{default_value}{None}}}{}
Liste de definition
\begin{quote}\begin{description}
\item[{Paramètres}] \leavevmode
\sphinxstyleliteralstrong{\sphinxupquote{code}} (\sphinxstyleliteralemphasis{\sphinxupquote{str}}) \textendash{} référence du de la liste

\item[{Renvoie}] \leavevmode
liste(s) de categories

\item[{Type renvoyé}] \leavevmode
dict

\end{description}\end{quote}

\end{fulllineitems}

\index{logs\_cfg() (méthode statique CFGBases)@\spxentry{logs\_cfg()}\spxextra{méthode statique CFGBases}}

\begin{fulllineitems}
\phantomsection\label{\detokenize{classes/cfgloader:toolbox.cfgmng.CFGBases.logs_cfg}}\pysiglinewithargsret{\sphinxbfcode{\sphinxupquote{static }}\sphinxbfcode{\sphinxupquote{logs\_cfg}}}{}{}
Configuration des logs
\begin{quote}\begin{description}
\item[{Exemple}] \leavevmode
\begin{sphinxVerbatim}[commandchars=\\\{\}]
\PYG{g+gp}{\PYGZgt{}\PYGZgt{}\PYGZgt{} }\PYG{k+kn}{import} \PYG{n+nn}{import} \PYG{n}{logging}\PYG{o}{.}\PYG{n}{config} \PYG{k}{as} \PYG{n}{log\PYGZus{}config}
\PYG{g+gp}{\PYGZgt{}\PYGZgt{}\PYGZgt{} }\PYG{k+kn}{import} \PYG{n+nn}{logging}
\PYG{g+gp}{\PYGZgt{}\PYGZgt{}\PYGZgt{} }\PYG{n}{log\PYGZus{}config}\PYG{o}{.}\PYG{n}{dictConfig}\PYG{p}{(}\PYG{n}{CFGBases}\PYG{o}{.}\PYG{n}{logs\PYGZus{}cfg}\PYG{p}{(}\PYG{p}{)}\PYG{p}{)}
\PYG{g+gp}{\PYGZgt{}\PYGZgt{}\PYGZgt{} }\PYG{n}{tracker} \PYG{o}{=} \PYG{n}{logging}\PYG{o}{.}\PYG{n}{getLogger}\PYG{p}{(}\PYG{l+s+s1}{\PYGZsq{}}\PYG{l+s+s1}{PROD|TEST}\PYG{l+s+s1}{\PYGZsq{}}\PYG{p}{)}
\PYG{g+gp}{\PYGZgt{}\PYGZgt{}\PYGZgt{} }\PYG{n}{tracker}\PYG{o}{.}\PYG{n}{info}\PYG{p}{(}\PYG{l+s+s2}{\PYGZdq{}}\PYG{l+s+s2}{Exemple dun message d}\PYG{l+s+s2}{\PYGZsq{}}\PYG{l+s+s2}{information}\PYG{l+s+s2}{\PYGZdq{}}\PYG{p}{)}
\end{sphinxVerbatim}

\end{description}\end{quote}

\end{fulllineitems}

\index{mailing\_lib() (méthode statique CFGBases)@\spxentry{mailing\_lib()}\spxextra{méthode statique CFGBases}}

\begin{fulllineitems}
\phantomsection\label{\detokenize{classes/cfgloader:toolbox.cfgmng.CFGBases.mailing_lib}}\pysiglinewithargsret{\sphinxbfcode{\sphinxupquote{static }}\sphinxbfcode{\sphinxupquote{mailing\_lib}}}{\emph{\DUrole{n}{code}}}{}
Mail préparé
\begin{quote}\begin{description}
\item[{Paramètres}] \leavevmode
\sphinxstyleliteralstrong{\sphinxupquote{code}} (\sphinxstyleliteralemphasis{\sphinxupquote{str}}) \textendash{} référence du mail à envoyer

\item[{Renvoie}] \leavevmode
mail

\end{description}\end{quote}

\end{fulllineitems}

\index{normalizor() (méthode statique CFGBases)@\spxentry{normalizor()}\spxextra{méthode statique CFGBases}}

\begin{fulllineitems}
\phantomsection\label{\detokenize{classes/cfgloader:toolbox.cfgmng.CFGBases.normalizor}}\pysiglinewithargsret{\sphinxbfcode{\sphinxupquote{static }}\sphinxbfcode{\sphinxupquote{normalizor}}}{}{}
Parametres de normalisation de formulaire
:return: parametres de normaisation
:rtype: dict

\end{fulllineitems}

\index{validator() (méthode statique CFGBases)@\spxentry{validator()}\spxextra{méthode statique CFGBases}}

\begin{fulllineitems}
\phantomsection\label{\detokenize{classes/cfgloader:toolbox.cfgmng.CFGBases.validator}}\pysiglinewithargsret{\sphinxbfcode{\sphinxupquote{static }}\sphinxbfcode{\sphinxupquote{validator}}}{}{}
Parametres de validation de formulaire
\begin{quote}\begin{description}
\item[{Paramètres}] \leavevmode
\sphinxstyleliteralstrong{\sphinxupquote{code}} (\sphinxstyleliteralemphasis{\sphinxupquote{str}}) \textendash{} référence du formulaire

\item[{Renvoie}] \leavevmode
parametres de validation

\item[{Type renvoyé}] \leavevmode
dict

\end{description}\end{quote}

\end{fulllineitems}


\end{fulllineitems}



\chapter{Gestions des logs}
\label{\detokenize{classes/cfgloader:gestions-des-logs}}\begin{itemize}
\item {} 
Configuration des logs

\item {} 
Traitement des erreurs et des exceptions

\end{itemize}

pathfile : tolbox/logmng.py

\phantomsection\label{\detokenize{classes/cfgloader:module-toolbox.logmng}}\index{module@\spxentry{module}!toolbox.logmng@\spxentry{toolbox.logmng}}\index{toolbox.logmng@\spxentry{toolbox.logmng}!module@\spxentry{module}}\index{CError@\spxentry{CError}}

\begin{fulllineitems}
\phantomsection\label{\detokenize{classes/cfgloader:toolbox.logmng.CError}}\pysiglinewithargsret{\sphinxbfcode{\sphinxupquote{exception }}\sphinxbfcode{\sphinxupquote{CError}}}{\emph{\DUrole{n}{message}}, \emph{\DUrole{n}{status}}, \emph{\DUrole{n}{title}\DUrole{o}{=}\DUrole{default_value}{\textquotesingle{}ERRCustom\textquotesingle{}}}}{}
Gestion des erreurs et traitement des exceptions personnalisées

\end{fulllineitems}



\section{Class CReponder}
\label{\detokenize{classes/cfgloader:class-creponder}}\index{CReponder (classe dans toolbox.logmng)@\spxentry{CReponder}\spxextra{classe dans toolbox.logmng}}

\begin{fulllineitems}
\phantomsection\label{\detokenize{classes/cfgloader:toolbox.logmng.CReponder}}\pysiglinewithargsret{\sphinxbfcode{\sphinxupquote{class }}\sphinxbfcode{\sphinxupquote{CReponder}}}{\emph{\DUrole{n}{status}\DUrole{o}{=}\DUrole{default_value}{200}}, \emph{\DUrole{n}{data}\DUrole{o}{=}\DUrole{default_value}{None}}, \emph{\DUrole{n}{msg}\DUrole{o}{=}\DUrole{default_value}{None}}}{}~\index{ok() (CReponder property)@\spxentry{ok()}\spxextra{CReponder property}}

\begin{fulllineitems}
\phantomsection\label{\detokenize{classes/cfgloader:toolbox.logmng.CReponder.ok}}\pysigline{\sphinxbfcode{\sphinxupquote{property }}\sphinxbfcode{\sphinxupquote{ok}}}
Renvoie le status de la réponse
\begin{quote}\begin{description}
\item[{Type renvoyé}] \leavevmode
bool

\end{description}\end{quote}

\end{fulllineitems}

\index{response() (CReponder property)@\spxentry{response()}\spxextra{CReponder property}}

\begin{fulllineitems}
\phantomsection\label{\detokenize{classes/cfgloader:toolbox.logmng.CReponder.response}}\pysigline{\sphinxbfcode{\sphinxupquote{property }}\sphinxbfcode{\sphinxupquote{response}}}~\begin{quote}\begin{description}
\item[{Renvoie}] \leavevmode
\{« message »: self.message, « data »: self.data, “status\_code”: self.\_status\}

\end{description}\end{quote}

\end{fulllineitems}


\end{fulllineitems}



\section{Class CTracker}
\label{\detokenize{classes/cfgloader:class-ctracker}}
Module de gestion des logs :
\begin{itemize}
\item {} 
Récupération des logs, traitement

\item {} 
Execution sécurisé

\end{itemize}
\index{CTracker (classe dans toolbox.logmng)@\spxentry{CTracker}\spxextra{classe dans toolbox.logmng}}

\begin{fulllineitems}
\phantomsection\label{\detokenize{classes/cfgloader:toolbox.logmng.CTracker}}\pysigline{\sphinxbfcode{\sphinxupquote{class }}\sphinxbfcode{\sphinxupquote{CTracker}}}~\index{alert\_tracking() (méthode statique CTracker)@\spxentry{alert\_tracking()}\spxextra{méthode statique CTracker}}

\begin{fulllineitems}
\phantomsection\label{\detokenize{classes/cfgloader:toolbox.logmng.CTracker.alert_tracking}}\pysiglinewithargsret{\sphinxbfcode{\sphinxupquote{static }}\sphinxbfcode{\sphinxupquote{alert\_tracking}}}{\emph{\DUrole{n}{msg}}, \emph{\DUrole{n}{title}}, \emph{\DUrole{n}{code}\DUrole{o}{=}\DUrole{default_value}{\textquotesingle{}\textquotesingle{}}}}{}
Message d’alerte (WARNING)
\begin{quote}\begin{description}
\item[{Paramètres}] \leavevmode\begin{itemize}
\item {} 
\sphinxstyleliteralstrong{\sphinxupquote{msg}} (\sphinxstyleliteralemphasis{\sphinxupquote{str}}) \textendash{} message à ecrire dans logs

\item {} 
\sphinxstyleliteralstrong{\sphinxupquote{title}} (\sphinxstyleliteralemphasis{\sphinxupquote{str}}) \textendash{} Titre ou référence associé au message

\item {} 
\sphinxstyleliteralstrong{\sphinxupquote{code}} \textendash{} Code numérique

\end{itemize}

\end{description}\end{quote}

\end{fulllineitems}

\index{config() (méthode statique CTracker)@\spxentry{config()}\spxextra{méthode statique CTracker}}

\begin{fulllineitems}
\phantomsection\label{\detokenize{classes/cfgloader:toolbox.logmng.CTracker.config}}\pysiglinewithargsret{\sphinxbfcode{\sphinxupquote{static }}\sphinxbfcode{\sphinxupquote{config}}}{\emph{\DUrole{n}{mode}\DUrole{o}{=}\DUrole{default_value}{\textquotesingle{}PROD\textquotesingle{}}}}{}
Initialisation du gestionnaire de log à partir de la configuration enregistré

\begin{sphinxadmonition}{warning}{Avertissement:}
La configuration doit être configurer dans le fichier \textless{}PROJECT\_DIR\textgreater{}/cdg/.log.yml
\end{sphinxadmonition}
\begin{quote}\begin{description}
\item[{Exemple}] \leavevmode
\begin{sphinxVerbatim}[commandchars=\\\{\}]
\PYG{g+gp}{\PYGZgt{}\PYGZgt{}\PYGZgt{} }\PYG{n}{CTracker}\PYG{o}{.}\PYG{n}{config}\PYG{p}{(}\PYG{p}{)}
\PYG{g+go}{Configuration mode PRODUCTION}
\PYG{g+gp}{\PYGZgt{}\PYGZgt{}\PYGZgt{} }\PYG{n}{CTracker}\PYG{o}{.}\PYG{n}{config}\PYG{p}{(}\PYG{l+s+s2}{\PYGZdq{}}\PYG{l+s+s2}{DEBUG}\PYG{l+s+s2}{\PYGZdq{}}\PYG{p}{)}
\PYG{g+go}{Configuration mode debug}
\end{sphinxVerbatim}

\end{description}\end{quote}

\end{fulllineitems}

\index{critical\_tracking() (méthode statique CTracker)@\spxentry{critical\_tracking()}\spxextra{méthode statique CTracker}}

\begin{fulllineitems}
\phantomsection\label{\detokenize{classes/cfgloader:toolbox.logmng.CTracker.critical_tracking}}\pysiglinewithargsret{\sphinxbfcode{\sphinxupquote{static }}\sphinxbfcode{\sphinxupquote{critical\_tracking}}}{\emph{\DUrole{n}{msg}}, \emph{\DUrole{n}{title}}, \emph{\DUrole{n}{code}\DUrole{o}{=}\DUrole{default_value}{\textquotesingle{}\textquotesingle{}}}}{}
Message dcritique (CRITIQUE)
\begin{quote}\begin{description}
\item[{Paramètres}] \leavevmode\begin{itemize}
\item {} 
\sphinxstyleliteralstrong{\sphinxupquote{msg}} (\sphinxstyleliteralemphasis{\sphinxupquote{str}}) \textendash{} message à ecrire dans logs

\item {} 
\sphinxstyleliteralstrong{\sphinxupquote{title}} (\sphinxstyleliteralemphasis{\sphinxupquote{str}}) \textendash{} Titre ou référence associé au message

\item {} 
\sphinxstyleliteralstrong{\sphinxupquote{code}} \textendash{} Code numérique

\end{itemize}

\end{description}\end{quote}

\end{fulllineitems}

\index{error\_tracking() (méthode statique CTracker)@\spxentry{error\_tracking()}\spxextra{méthode statique CTracker}}

\begin{fulllineitems}
\phantomsection\label{\detokenize{classes/cfgloader:toolbox.logmng.CTracker.error_tracking}}\pysiglinewithargsret{\sphinxbfcode{\sphinxupquote{static }}\sphinxbfcode{\sphinxupquote{error\_tracking}}}{\emph{\DUrole{n}{msg}}, \emph{\DUrole{n}{title}}, \emph{\DUrole{n}{code}\DUrole{o}{=}\DUrole{default_value}{500}}}{}
Message d’error (ERROR)
\begin{quote}\begin{description}
\item[{Paramètres}] \leavevmode\begin{itemize}
\item {} 
\sphinxstyleliteralstrong{\sphinxupquote{msg}} (\sphinxstyleliteralemphasis{\sphinxupquote{str}}) \textendash{} message à ecrire dans logs

\item {} 
\sphinxstyleliteralstrong{\sphinxupquote{title}} (\sphinxstyleliteralemphasis{\sphinxupquote{str}}) \textendash{} Titre ou référence associé au message

\item {} 
\sphinxstyleliteralstrong{\sphinxupquote{code}} (\sphinxstyleliteralemphasis{\sphinxupquote{int}}) \textendash{} Code numérique

\end{itemize}

\end{description}\end{quote}

\end{fulllineitems}

\index{exception\_tracking() (méthode statique CTracker)@\spxentry{exception\_tracking()}\spxextra{méthode statique CTracker}}

\begin{fulllineitems}
\phantomsection\label{\detokenize{classes/cfgloader:toolbox.logmng.CTracker.exception_tracking}}\pysiglinewithargsret{\sphinxbfcode{\sphinxupquote{static }}\sphinxbfcode{\sphinxupquote{exception\_tracking}}}{\emph{\DUrole{n}{ex}}, \emph{\DUrole{n}{title}}}{}
Récupération et traitement des exceptions
\begin{quote}\begin{description}
\item[{Paramètres}] \leavevmode\begin{itemize}
\item {} 
\sphinxstyleliteralstrong{\sphinxupquote{ex}} \textendash{} Exception

\item {} 
\sphinxstyleliteralstrong{\sphinxupquote{title}} (\sphinxstyleliteralemphasis{\sphinxupquote{str}}) \textendash{} Information

\end{itemize}

\end{description}\end{quote}

\end{fulllineitems}

\index{flag() (méthode statique CTracker)@\spxentry{flag()}\spxextra{méthode statique CTracker}}

\begin{fulllineitems}
\phantomsection\label{\detokenize{classes/cfgloader:toolbox.logmng.CTracker.flag}}\pysiglinewithargsret{\sphinxbfcode{\sphinxupquote{static }}\sphinxbfcode{\sphinxupquote{flag}}}{\emph{\DUrole{n}{trace}}}{}
Permet de pointer la dernier action
\begin{quote}\begin{description}
\item[{Paramètres}] \leavevmode
\sphinxstyleliteralstrong{\sphinxupquote{trace}} (\sphinxstyleliteralemphasis{\sphinxupquote{str}}) \textendash{} Action à enregistrer

\end{description}\end{quote}

\end{fulllineitems}

\index{fntracker() (méthode statique CTracker)@\spxentry{fntracker()}\spxextra{méthode statique CTracker}}

\begin{fulllineitems}
\phantomsection\label{\detokenize{classes/cfgloader:toolbox.logmng.CTracker.fntracker}}\pysiglinewithargsret{\sphinxbfcode{\sphinxupquote{static }}\sphinxbfcode{\sphinxupquote{fntracker}}}{\emph{\DUrole{n}{fn}}, \emph{\DUrole{n}{action}}, \emph{\DUrole{o}{*}\DUrole{n}{args}}, \emph{\DUrole{o}{**}\DUrole{n}{kwargs}}}{}
Execution « securisé » d’une fonction avec gestions des erreurs
\begin{quote}\begin{description}
\item[{Paramètres}] \leavevmode\begin{itemize}
\item {} 
\sphinxstyleliteralstrong{\sphinxupquote{fn}} \textendash{} fonction a executer

\item {} 
\sphinxstyleliteralstrong{\sphinxupquote{action}} \textendash{} Titre de l’execution pour tracabilité

\item {} 
\sphinxstyleliteralstrong{\sphinxupquote{args}} \textendash{} argument de la fonction

\item {} 
\sphinxstyleliteralstrong{\sphinxupquote{kwargs}} \textendash{} parametres supplementaire (status par defaut en cas de reussite)

\end{itemize}

\item[{Type renvoyé}] \leavevmode
{\hyperref[\detokenize{classes/cfgloader:toolbox.logmng.CReponder}]{\sphinxcrossref{CReponder}}}

\item[{Exemple}] \leavevmode
\begin{sphinxVerbatim}[commandchars=\\\{\}]
\PYG{g+gp}{\PYGZgt{}\PYGZgt{}\PYGZgt{} }\PYG{k+kn}{from} \PYG{n+nn}{toolbox}\PYG{n+nn}{.}\PYG{n+nn}{logmng} \PYG{k+kn}{import} \PYG{n}{CTracker}
\PYG{g+gp}{\PYGZgt{}\PYGZgt{}\PYGZgt{} }\PYG{k}{def} \PYG{n+nf}{fn}\PYG{p}{(}\PYG{n}{param}\PYG{p}{)}\PYG{p}{:}\PYG{p}{:}
\PYG{g+gp}{\PYGZgt{}\PYGZgt{}\PYGZgt{} }    \PYG{k}{return} \PYG{n+nb}{int}\PYG{p}{(}\PYG{n}{param}\PYG{p}{)}
\end{sphinxVerbatim}

\begin{sphinxVerbatim}[commandchars=\\\{\}]
\PYG{g+gp}{\PYGZgt{}\PYGZgt{}\PYGZgt{} }\PYG{n}{r} \PYG{o}{=} \PYG{n}{CTracker}\PYG{o}{.}\PYG{n}{fntracker}\PYG{p}{(}\PYG{n}{fn}\PYG{p}{,} \PYG{l+s+s1}{\PYGZsq{}}\PYG{l+s+s1}{Test de convertion int}\PYG{l+s+s1}{\PYGZsq{}}\PYG{p}{,} \PYG{l+s+s1}{\PYGZsq{}}\PYG{l+s+s1}{j}\PYG{l+s+s1}{\PYGZsq{}}\PYG{p}{)}
\PYG{g+gp}{\PYGZgt{}\PYGZgt{}\PYGZgt{} }\PYG{n}{r}\PYG{o}{.}\PYG{n}{response}
\PYG{g+go}{\PYGZob{}\PYGZsq{}message\PYGZsq{}: \PYGZdq{}invalid literal for int() with base 10: \PYGZsq{}j\PYGZsq{}\PYGZdq{}, \PYGZsq{}data\PYGZsq{}: None, \PYGZsq{}status\PYGZus{}code\PYGZsq{}: 500\PYGZcb{}}
\end{sphinxVerbatim}

\begin{sphinxVerbatim}[commandchars=\\\{\}]
\PYG{g+gp}{\PYGZgt{}\PYGZgt{}\PYGZgt{} }\PYG{n}{r} \PYG{o}{=} \PYG{n}{CTracker}\PYG{o}{.}\PYG{n}{fntracker}\PYG{p}{(}\PYG{n}{fn}\PYG{p}{,} \PYG{l+s+s1}{\PYGZsq{}}\PYG{l+s+s1}{Test de convertion int}\PYG{l+s+s1}{\PYGZsq{}}\PYG{p}{,} \PYG{l+s+s1}{\PYGZsq{}}\PYG{l+s+s1}{589321}\PYG{l+s+s1}{\PYGZsq{}}\PYG{p}{)}
\PYG{g+gp}{\PYGZgt{}\PYGZgt{}\PYGZgt{} }\PYG{n}{r}\PYG{o}{.}\PYG{n}{response}
\PYG{g+go}{\PYGZob{}\PYGZsq{}message\PYGZsq{}: None, \PYGZsq{}data\PYGZsq{}: 589321, \PYGZsq{}status\PYGZus{}code\PYGZsq{}: 200\PYGZcb{}}
\end{sphinxVerbatim}

\end{description}\end{quote}

\end{fulllineitems}

\index{info\_tracking() (méthode statique CTracker)@\spxentry{info\_tracking()}\spxextra{méthode statique CTracker}}

\begin{fulllineitems}
\phantomsection\label{\detokenize{classes/cfgloader:toolbox.logmng.CTracker.info_tracking}}\pysiglinewithargsret{\sphinxbfcode{\sphinxupquote{static }}\sphinxbfcode{\sphinxupquote{info\_tracking}}}{\emph{\DUrole{n}{msg}}, \emph{\DUrole{n}{title}}, \emph{\DUrole{n}{code}\DUrole{o}{=}\DUrole{default_value}{\textquotesingle{}\textquotesingle{}}}}{}
Message d’info (INFO)
\begin{quote}\begin{description}
\item[{Paramètres}] \leavevmode\begin{itemize}
\item {} 
\sphinxstyleliteralstrong{\sphinxupquote{msg}} (\sphinxstyleliteralemphasis{\sphinxupquote{str}}) \textendash{} message à ecrire dans logs

\item {} 
\sphinxstyleliteralstrong{\sphinxupquote{title}} (\sphinxstyleliteralemphasis{\sphinxupquote{str}}) \textendash{} Titre ou référence associé au message

\item {} 
\sphinxstyleliteralstrong{\sphinxupquote{code}} \textendash{} Code numérique

\end{itemize}

\end{description}\end{quote}

\end{fulllineitems}

\index{msg\_tracking() (méthode statique CTracker)@\spxentry{msg\_tracking()}\spxextra{méthode statique CTracker}}

\begin{fulllineitems}
\phantomsection\label{\detokenize{classes/cfgloader:toolbox.logmng.CTracker.msg_tracking}}\pysiglinewithargsret{\sphinxbfcode{\sphinxupquote{static }}\sphinxbfcode{\sphinxupquote{msg\_tracking}}}{\emph{\DUrole{n}{msg}}, \emph{\DUrole{n}{title}}, \emph{\DUrole{n}{log\_level}\DUrole{o}{=}\DUrole{default_value}{20}}, \emph{\DUrole{n}{code}\DUrole{o}{=}\DUrole{default_value}{0}}}{}
Tracking message
\begin{quote}\begin{description}
\item[{Paramètres}] \leavevmode\begin{itemize}
\item {} 
\sphinxstyleliteralstrong{\sphinxupquote{msg}} (\sphinxstyleliteralemphasis{\sphinxupquote{str}}) \textendash{} message à ecrire dans logs

\item {} 
\sphinxstyleliteralstrong{\sphinxupquote{title}} (\sphinxstyleliteralemphasis{\sphinxupquote{str}}) \textendash{} Titre ou référence associé au message

\item {} 
\sphinxstyleliteralstrong{\sphinxupquote{log\_level}} (\sphinxstyleliteralemphasis{\sphinxupquote{int}}) \textendash{} LOG LEVEL Niveau de l’alert (DEBUG | INFO | WARN | )

\item {} 
\sphinxstyleliteralstrong{\sphinxupquote{code}} (\sphinxstyleliteralemphasis{\sphinxupquote{int}}) \textendash{} Code numérique

\end{itemize}

\end{description}\end{quote}

\end{fulllineitems}


\end{fulllineitems}



\chapter{Module de validation}
\label{\detokenize{classes/cfgloader:module-de-validation}}
Validation de formulaire.

\begin{sphinxadmonition}{note}{Note:}
Les shémas de validation sont sauvegarder dans le dossier de configuration validators
Les schéma de normalization dans normalizator
\end{sphinxadmonition}
\phantomsection\label{\detokenize{classes/cfgloader:module-toolbox.validata}}\index{module@\spxentry{module}!toolbox.validata@\spxentry{toolbox.validata}}\index{toolbox.validata@\spxentry{toolbox.validata}!module@\spxentry{module}}\index{Validata (classe dans toolbox.validata)@\spxentry{Validata}\spxextra{classe dans toolbox.validata}}

\begin{fulllineitems}
\phantomsection\label{\detokenize{classes/cfgloader:toolbox.validata.Validata}}\pysiglinewithargsret{\sphinxbfcode{\sphinxupquote{class }}\sphinxbfcode{\sphinxupquote{Validata}}}{\emph{\DUrole{n}{scheme}}, \emph{\DUrole{o}{*}\DUrole{n}{args}}, \emph{\DUrole{o}{**}\DUrole{n}{kwargs}}}{}
Validators
\index{check\_post\_data() (méthode statique Validata)@\spxentry{check\_post\_data()}\spxextra{méthode statique Validata}}

\begin{fulllineitems}
\phantomsection\label{\detokenize{classes/cfgloader:toolbox.validata.Validata.check_post_data}}\pysiglinewithargsret{\sphinxbfcode{\sphinxupquote{static }}\sphinxbfcode{\sphinxupquote{check\_post\_data}}}{\emph{\DUrole{n}{data}}, \emph{\DUrole{n}{form\_ref}}}{}
verification donnees formulaire recu
\begin{quote}\begin{description}
\item[{Paramètres}] \leavevmode\begin{itemize}
\item {} 
\sphinxstyleliteralstrong{\sphinxupquote{data}} (\sphinxstyleliteralemphasis{\sphinxupquote{dict}}) \textendash{} formulaire de données

\item {} 
\sphinxstyleliteralstrong{\sphinxupquote{form\_ref}} (\sphinxstyleliteralemphasis{\sphinxupquote{str}}) \textendash{} reference validateur

\end{itemize}

\item[{Renvoie}] \leavevmode
formulaire traité

\end{description}\end{quote}

\end{fulllineitems}

\index{normalisation() (méthode Validata)@\spxentry{normalisation()}\spxextra{méthode Validata}}

\begin{fulllineitems}
\phantomsection\label{\detokenize{classes/cfgloader:toolbox.validata.Validata.normalisation}}\pysiglinewithargsret{\sphinxbfcode{\sphinxupquote{normalisation}}}{\emph{\DUrole{n}{document}}}{}~\begin{quote}\begin{description}
\item[{Paramètres}] \leavevmode\begin{itemize}
\item {} 
\sphinxstyleliteralstrong{\sphinxupquote{document}} \textendash{} 

\item {} 
\sphinxstyleliteralstrong{\sphinxupquote{args}} \textendash{} 

\item {} 
\sphinxstyleliteralstrong{\sphinxupquote{kwargs}} \textendash{} 

\end{itemize}

\item[{Renvoie}] \leavevmode


\end{description}\end{quote}

\end{fulllineitems}

\index{validation() (méthode Validata)@\spxentry{validation()}\spxextra{méthode Validata}}

\begin{fulllineitems}
\phantomsection\label{\detokenize{classes/cfgloader:toolbox.validata.Validata.validation}}\pysiglinewithargsret{\sphinxbfcode{\sphinxupquote{validation}}}{\emph{\DUrole{n}{document}}, \emph{\DUrole{o}{*}\DUrole{n}{args}}, \emph{\DUrole{o}{**}\DUrole{n}{kwargs}}}{}~\begin{quote}\begin{description}
\item[{Paramètres}] \leavevmode\begin{itemize}
\item {} 
\sphinxstyleliteralstrong{\sphinxupquote{document}} \textendash{} 

\item {} 
\sphinxstyleliteralstrong{\sphinxupquote{args}} \textendash{} 

\item {} 
\sphinxstyleliteralstrong{\sphinxupquote{kwargs}} \textendash{} 

\end{itemize}

\item[{Renvoie}] \leavevmode


\end{description}\end{quote}

\end{fulllineitems}


\end{fulllineitems}



\chapter{Gestion de mailing}
\label{\detokenize{classes/cfgloader:module-toolbox.mailbot}}\label{\detokenize{classes/cfgloader:gestion-de-mailing}}\index{module@\spxentry{module}!toolbox.mailbot@\spxentry{toolbox.mailbot}}\index{toolbox.mailbot@\spxentry{toolbox.mailbot}!module@\spxentry{module}}
Module de Gestion de mail préparés

pathfile : toolbox/mailbot.py


\section{Pré\sphinxhyphen{}Requis}
\label{\detokenize{classes/cfgloader:pre-requis}}
\begin{sphinxadmonition}{warning}{Avertissement:}
Indiquer les parametres smtp dans le fichiers de configuration \textless{}PROJECT\_NAME\textgreater{}/cfg/.app.yml
\end{sphinxadmonition}

\begin{sphinxVerbatim}[commandchars=\\\{\}]
\PYG{n+nt}{smtp}\PYG{p}{:}
 \PYG{n+nt}{h}\PYG{p}{:} \PYG{l+lScalar+lScalarPlain}{smtp\PYGZhy{}host\PYGZus{}adresse}
 \PYG{n+nt}{po}\PYG{p}{:} \PYG{l+lScalar+lScalarPlain}{port\PYGZus{}smtp}
 \PYG{n+nt}{m}\PYG{p}{:} \PYG{l+lScalar+lScalarPlain}{mail\PYGZus{}authen}
 \PYG{n+nt}{pw}\PYG{p}{:} \PYG{l+lScalar+lScalarPlain}{password\PYGZus{}auth}
 \PYG{n+nt}{h\PYGZus{}s }\PYG{p}{:} \PYG{l+lScalar+lScalarPlain}{name\PYGZus{}sender}\PYG{l+lScalar+lScalarPlain}{ }\PYG{l+lScalar+lScalarPlain}{\PYGZlt{}email\PYGZgt{}}
\end{sphinxVerbatim}

\begin{sphinxadmonition}{warning}{Avertissement:}
Les mails sont à définir dans le ficchier \textless{}PROJECT\_NAME\textgreater{}/cfg/mailing.yml au format suivant
\end{sphinxadmonition}

\begin{sphinxVerbatim}[commandchars=\\\{\}]
\PYG{n+nt}{footer}\PYG{p}{:}
 \PYG{n+nt}{html}\PYG{p}{:} \PYG{l+lScalar+lScalarPlain}{\PYGZlt{}Pied}\PYG{l+lScalar+lScalarPlain}{ }\PYG{l+lScalar+lScalarPlain}{de}\PYG{l+lScalar+lScalarPlain}{ }\PYG{l+lScalar+lScalarPlain}{mail}\PYG{l+lScalar+lScalarPlain}{ }\PYG{l+lScalar+lScalarPlain}{unique}\PYG{l+lScalar+lScalarPlain}{ }\PYG{l+lScalar+lScalarPlain}{pour}\PYG{l+lScalar+lScalarPlain}{ }\PYG{l+lScalar+lScalarPlain}{tous}\PYG{l+lScalar+lScalarPlain}{ }\PYG{l+lScalar+lScalarPlain}{les}\PYG{l+lScalar+lScalarPlain}{ }\PYG{l+lScalar+lScalarPlain}{mails}\PYG{l+lScalar+lScalarPlain}{ }\PYG{l+lScalar+lScalarPlain}{(signature,}\PYG{l+lScalar+lScalarPlain}{ }\PYG{l+lScalar+lScalarPlain}{rgpd...)\PYGZgt{}}
 \PYG{n+nt}{text}\PYG{p}{:} \PYG{l+lScalar+lScalarPlain}{\PYGZlt{}Pied}\PYG{l+lScalar+lScalarPlain}{ }\PYG{l+lScalar+lScalarPlain}{de}\PYG{l+lScalar+lScalarPlain}{ }\PYG{l+lScalar+lScalarPlain}{mail}\PYG{l+lScalar+lScalarPlain}{ }\PYG{l+lScalar+lScalarPlain}{unique}\PYG{l+lScalar+lScalarPlain}{ }\PYG{l+lScalar+lScalarPlain}{pour}\PYG{l+lScalar+lScalarPlain}{ }\PYG{l+lScalar+lScalarPlain}{tous}\PYG{l+lScalar+lScalarPlain}{ }\PYG{l+lScalar+lScalarPlain}{les}\PYG{l+lScalar+lScalarPlain}{ }\PYG{l+lScalar+lScalarPlain}{mails}\PYG{l+lScalar+lScalarPlain}{ }\PYG{l+lScalar+lScalarPlain}{(signature,}\PYG{l+lScalar+lScalarPlain}{ }\PYG{l+lScalar+lScalarPlain}{rgpd...)\PYGZgt{}}
\PYG{n+nt}{code\PYGZus{}mail}\PYG{p}{:}
 \PYG{n+nt}{html}\PYG{p}{:} \PYG{l+lScalar+lScalarPlain}{\PYGZlt{}ici}\PYG{l+lScalar+lScalarPlain}{ }\PYG{l+lScalar+lScalarPlain}{mail}\PYG{l+lScalar+lScalarPlain}{ }\PYG{l+lScalar+lScalarPlain}{au}\PYG{l+lScalar+lScalarPlain}{ }\PYG{l+lScalar+lScalarPlain}{format}\PYG{l+lScalar+lScalarPlain}{ }\PYG{l+lScalar+lScalarPlain}{HTML\PYGZgt{}}
 \PYG{n+nt}{text }\PYG{p}{:} \PYG{l+lScalar+lScalarPlain}{\PYGZlt{}Le}\PYG{l+lScalar+lScalarPlain}{ }\PYG{l+lScalar+lScalarPlain}{mail}\PYG{l+lScalar+lScalarPlain}{ }\PYG{l+lScalar+lScalarPlain}{au}\PYG{l+lScalar+lScalarPlain}{ }\PYG{l+lScalar+lScalarPlain}{format}\PYG{l+lScalar+lScalarPlain}{ }\PYG{l+lScalar+lScalarPlain}{texte\PYGZgt{}}
 \PYG{n+nt}{objt }\PYG{p}{:} \PYG{l+lScalar+lScalarPlain}{\PYGZlt{}Objet}\PYG{l+lScalar+lScalarPlain}{ }\PYG{l+lScalar+lScalarPlain}{du}\PYG{l+lScalar+lScalarPlain}{ }\PYG{l+lScalar+lScalarPlain}{mail\PYGZgt{}}
\end{sphinxVerbatim}


\section{Class CMailer}
\label{\detokenize{classes/cfgloader:class-cmailer}}\index{CMailer (classe dans toolbox.mailbot)@\spxentry{CMailer}\spxextra{classe dans toolbox.mailbot}}

\begin{fulllineitems}
\phantomsection\label{\detokenize{classes/cfgloader:toolbox.mailbot.CMailer}}\pysigline{\sphinxbfcode{\sphinxupquote{class }}\sphinxbfcode{\sphinxupquote{CMailer}}}~\index{presend() (méthode CMailer)@\spxentry{presend()}\spxextra{méthode CMailer}}

\begin{fulllineitems}
\phantomsection\label{\detokenize{classes/cfgloader:toolbox.mailbot.CMailer.presend}}\pysiglinewithargsret{\sphinxbfcode{\sphinxupquote{presend}}}{\emph{\DUrole{n}{code}}, \emph{\DUrole{n}{name}\DUrole{o}{=}\DUrole{default_value}{\textquotesingle{}\textquotesingle{}}}, \emph{\DUrole{o}{**}\DUrole{n}{data\_field}}}{}
Preparation pour envoi d’un message mail
\begin{quote}\begin{description}
\item[{Paramètres}] \leavevmode\begin{itemize}
\item {} 
\sphinxstyleliteralstrong{\sphinxupquote{email}} (\sphinxstyleliteralemphasis{\sphinxupquote{str}}) \textendash{} email destinataire

\item {} 
\sphinxstyleliteralstrong{\sphinxupquote{code}} (\sphinxstyleliteralemphasis{\sphinxupquote{str}}) \textendash{} réfénce du mail à chargé

\item {} 
\sphinxstyleliteralstrong{\sphinxupquote{name}} (\sphinxstyleliteralemphasis{\sphinxupquote{str}}) \textendash{} nom du destinataire

\item {} 
\sphinxstyleliteralstrong{\sphinxupquote{data\_field}} (\sphinxstyleliteralemphasis{\sphinxupquote{dict}}) \textendash{} liste de données relatif à des champs définis dans le mails

\end{itemize}

\end{description}\end{quote}

\end{fulllineitems}


\end{fulllineitems}



\chapter{Indices and tables}
\label{\detokenize{index:indices-and-tables}}\begin{itemize}
\item {} 
\DUrole{xref,std,std-ref}{genindex}

\item {} 
\DUrole{xref,std,std-ref}{modindex}

\item {} 
\DUrole{xref,std,std-ref}{search}

\end{itemize}


\renewcommand{\indexname}{Index des modules Python}
\begin{sphinxtheindex}
\let\bigletter\sphinxstyleindexlettergroup
\bigletter{t}
\item\relax\sphinxstyleindexentry{toolbox.cfgmng}\sphinxstyleindexpageref{classes/cfgloader:\detokenize{module-toolbox.cfgmng}}
\item\relax\sphinxstyleindexentry{toolbox.dtemng}\sphinxstyleindexpageref{modules/dtemng:\detokenize{module-toolbox.dtemng}}
\item\relax\sphinxstyleindexentry{toolbox.features}\sphinxstyleindexpageref{modules/tools:\detokenize{module-toolbox.features}}
\item\relax\sphinxstyleindexentry{toolbox.logmng}\sphinxstyleindexpageref{classes/cfgloader:\detokenize{module-toolbox.logmng}}
\item\relax\sphinxstyleindexentry{toolbox.mailbot}\sphinxstyleindexpageref{classes/cfgloader:\detokenize{module-toolbox.mailbot}}
\item\relax\sphinxstyleindexentry{toolbox.tools}\sphinxstyleindexpageref{modules/tools:\detokenize{module-toolbox.tools}}
\item\relax\sphinxstyleindexentry{toolbox.validata}\sphinxstyleindexpageref{classes/cfgloader:\detokenize{module-toolbox.validata}}
\end{sphinxtheindex}

\renewcommand{\indexname}{Index}
\printindex
\end{document}